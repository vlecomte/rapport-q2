\subsection*{Introduction}

Dans de nombreux domaines de la physique,
on retrouve des phénomènes périodiques.
Une manière de simplifier leur étude
est de ramener les fonctions périodiques à une somme de sinusoïdales,
puis d'étudier ces dernières avec la méthode des phaseurs.
\begin{itemize}
    \item La première étape consiste à prendre la transformée de Fourier,
        de la fonction périodique quelconque donnée.
        Toutefois, elle demande un développement d'analyse assez important.
        Nous n'étudierons ici que des signaux d'entrée sinusoïdaux purs
        (qui correspondent d'ailleurs à des sons purs).
    \item La deuxième étape consiste à représenter chacune des sinusoïdales
        sous une forme qui simplifie les calculs classiques à effectuer,
        c'est-à-dire la résolution d'équations différentielles,
        puis plus spécifiquement la résolution de circuits électriques.
\end{itemize}

Dans le cadre de ce rapport,
nous utilisons les phaseurs pour modéliser les filtres de fréquences,
les mouvements de la membrane,
et une version simplifiée de la propagation du son en une dimension.

Dans ce chapitre, nous allons partir d'un exemple simple de modélisation
où la méthode des phaseurs peut simplifier les calculs,
puis nous allons introduire la notion de phaseurs
en général et ses propriétés
intéressantes dans la modélisation de circuits électriques en particulier.

\subsubsection*{Plan du chapitre}
\begin{enumerate}
    \item Nous commencerons par illustrer la \emph{méthode classique}
        de résolution en prenant l'exemple de la modélisation
        du filtre passe-bas.
    \item Puis nous définirons le concept de phaseurs, 
        en tant qu'\emph{isomorphisme} et étudierons
        ses propriétés fondamentales.
    \item Ensuite, nous observerons comment la \emph{différentiation}
        s'exprime pour les phaseurs, et ses conséquences sur
        la recherche de solutions particulières sinusoïdales
        de certaines équations différentielles.
    \item Après cela, nous représenterons la \emph{moyenne du produit}
        de deux fonctions sinusoïdales avec leurs phaseurs.
    \item Enfin, nous présenterons la notion d'\emph{impédance complexe}
        et la manière dont elle simplifie les calculs
        dans des circuits en régime sinusoïdal stable.
\end{enumerate}
