\section{Impédance complexe}

La dernière notion liée aux phaseurs que nous allons introduire
et celle d'impédance complexe.
La méthode de la section~\ref{sec:phaseurs/diff} permet déjà
de simplifier fortement les calculs pour les circuits.
Toutefois, il y a moyen d'aller plus loin.

Pour rappel, la représentation sous forme de phaseurs est une application
linéaire,
c'est-à-dire qu'elle conserve les sommes et les multiplications par un scalaire.
Dans le cadre d'un circuit, cela signifie
que les lois de Kirchhoff et la loi d'Ohm sont conservées,
et peuvent être utilisées de la même manière que pour des fonctions
quelconques.

Toutefois, cela ne permet pas a priori de modéliser les capacités
et inductances directement.
C'est pourquoi nous définissons les impédances, qui sont une généralisation
des résistances.

Une résistance est définie par la loi d'Ohm:
\begin{equation}
    R = \frac{v(t)}{i(t)} = \frac{V}{I}
\end{equation}
à condition que la tension et le courant soient en phase.

De la même manière on peut partir de l'équation caractéristique
d'une inductance:
\begin{equation}
    v(t) = L\frac{di(t)}{dt}
\end{equation}
et la passer sous forme de phaseurs:
\begin{equation}
    V = L (j\omega) I
\end{equation}
pour définir l'impédance $X_L$ de cette inductance:
\begin{equation}
    X_L = \frac{V}{I} = j\omega L
\end{equation}

Et pour une capacité:
\begin{equation}
    i(t) = C\frac{dv(t)}{dt}
\end{equation}
se réécrit:
\begin{equation}
    I = C (j\omega) V
\end{equation}
pour définir l'impédance $X_C$ de cette capacité:
\begin{equation}
    \frac{V}{I} = \frac{1}{j\omega C}
\end{equation}

Cela nous permet de définir une loi d'Ohm généralisée:
\begin{equation}
    V = XI
\end{equation}
où $X$ est l'impédance du composant.
\begin{itemize}
    \item Pour une résistance $R$, $X_R=R$.
    \item Pour une inductance $L$, $X_L=j\omega L$.
    \item Pour une capacité $C$, $X_C=1/j\omega C$.
\end{itemize}
