\section{Différentiation}
\label{sec:phaseurs/diff}

Nous allons maintenant déterminer l'opération qu'il faut effectuer
sur le phaseur d'une fonction pour obtenir le phaseur de sa dérivée temporelle.

Prenons comme d'habitude une fonction $f(t) = |F|\cos(\omega t + \phi)$
de phaseur $F = |F|e^{j\phi}$
et dérivons-la:
\begin{equation}
    \frac{d}{dt}f(t) = \omega (-|F|\sin(\omega t + \phi))
    = \omega |F| \cos(\omega t + \pi/2 + \phi)
\end{equation}

Le phaseur correspondant $F'$ vaut:
\begin{equation}
    F' = \omega |F| e^{j(\pi/2 + \phi)} = \omega |F| e^{j\pi/2} e^{j\phi}
    = j\omega |F| e^{j\phi} = j\omega F
\end{equation}

On découvre donc que quand on dérive une fonction sinusoïdale,
son phaseur est simplement multiplié par $j\omega$.
\footnote{
    Alternativement, on aurait pu voir que
    $(\mathfrak{Re}\{ |F|e^{j\phi} \cdot e^{j\omega t}  \})'
    = \mathfrak{Re}\{(|F|e^{j\phi} \cdot e^{j\omega t})'\}
    = \mathfrak{Re}\{j\omega |F| e^{j\phi} \cdot e^{j\omega t}\}$,
    ce qui montre également que le phaseur $|F|e^{j\phi}$
    est multiplié par $j\omega$.
}
Cela a comme conséquence immédiate de simplifier énormément
le calcul de solutions particulières sinusoïdales d'équations différentielles,
les transformant en de simples équations linéaires sur les complexes.

Reprenons l'équation différentielle décrivant le filtre passe-bas
\eqref{eq:diff-passe-bas}:
\[
    v\ind{in} = RC\frac{dv_C}{dt} + v_C
\]
on peut la réécrire sous forme de phaseurs:
\begin{equation}
    V\ind{in} = RC(j\omega)V_C + V_C
\end{equation}
ce qui donne immédiatement la solution:
\begin{equation}
    V_C = \frac{V\ind{in}}{1 + j\omega RC}
\end{equation}

On peut ensuite extraire l'amplitude et le déphasage de $v_C$
en calculant respectivement le module et l'argument de $\overline{V_C}$:
\begin{equation}
    \left\{
        \begin{array}{ccl}
            |V_C| &=& |V\ind{in}|\,/\,|1+j\omega RC|
            = |V\ind{in}|\,/\sqrt{1 + (\omega RC)^2} \\
            \phi &=& \arg(V_C) = -\arg(1+j\omega RC)
            = - \arctan(\omega RC)
        \end{array}
    \right.
\end{equation}
ce qui correspond exactement aux résultats trouvés précédemment.
