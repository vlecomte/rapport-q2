\section{Moyenne d'un produit}

Le produit de deux fonctions $f$ et $g$ dans $\mathbb{S}_\omega$
n'est pas interne.
En effet, la fréquence de $f \cdot g$ est doublée,
et une constante s'ajoute à la sinusoïde.

Toutefois, il est possible d'exprimer de manière intéressante
la moyenne de ce produit.
Un exemple d'application est la puissance moyenne absorbée par un circuit,
qui vaut la moyenne du produit $v\cdot i$ de la tension et du courant.

Posons $f(t) = F\cos(\omega t + \phi)$ et $g(t) = G\cos(\omega t + \psi)$.
Exprimons le produit de ces deux fonctions:
\begin{equation}
    \begin{split}
        (f\cdot g)(t) &= FG\cos(\omega t + \phi)\cos(\omega t + \psi) \\
        &= \frac{FG}{2}(\cos(2\omega t + \phi + \psi) + \cos(\phi - \psi))
    \end{split}
\end{equation}
la seconde égalité découlant de la formule de Simpson.

Le terme $\cos(2\omega t + \phi + \psi)$ oscille autour de 0,
donc sa contribution à la moyenne est nulle.
Il reste donc le terme constant $\frac{1}{2}FG\cos(\phi - \psi)$.
Nous reconnaissons ici la forme du produit scalaire des phaseurs
$Fe^{j\phi}$ et $Ge^{j\psi}$ par rapport à la base orthonormée $\{1,j\}$,
divisée par deux.

Exprimons ce produit scalaire pour s'en assurer:
\begin{equation}
    \begin{array}{rcl}
        \scal{Fe^{j\phi}}{Ge^{j\psi}}
        &=& \scal{F(\cos\phi+j\sin\phi)}{G(\cos\psi+j\sin\psi)} \\
        &=& FG\,\big[\cos\phi\cos\psi \scal{1}{1}
        + \cos\phi\sin\psi \scal{1}{j} \\
        && + \sin\phi\cos\psi \scal{j}{1}
        + \sin\phi\sin\psi \scal{j}{j}\big] \\
        &=& FG(\cos\phi\cos\psi + \sin\phi\sin\psi) \\
        &=& FG\cos(\phi-\psi)
    \end{array}
\end{equation}

Nous avons donc prouvé que la moyenne de $(f\cdot g)(t)$
vaut $\frac{1}{2}\scal{\overline{F}}{\overline{G}}$.
Cela permet notamment de tirer avantage de la forme classique $a+bj$
des complexes pour les phaseurs,
avec laquelle $\scal{a+bj}{c+dj}$ vaut simplement $ac + bd$.
