\section{Moyenne d'un produit}

Le produit de deux fonctions $f$ et $g$ dans $\mathbb{S}_\omega$
n'est pas interne.
En effet, la fréquence de $f \cdot g$ est doublée,
et une constante s'ajoute à la sinusoïde.

Toutefois, il est possible d'exprimer de manière intéressante
la moyenne de ce produit.
Un exemple d'application est la puissance moyenne absorbée par un circuit,
qui vaut la moyenne du produit $v\cdot i$ de la tension et du courant.

Posons $f(t) = F\cos(\omega t + \phi)$ et $g(t) = G\cos(\omega t + \psi)$.
Exprimons le produit de ces deux fonctions:
\begin{equation}
    \begin{split}
        (f\cdot g)(t) &= FG\cos(\omega t + \phi)\cos(\omega t + \psi) \\
        &= \frac{FG}{2}(\cos(2\omega t + \phi + \psi) + \cos(\phi - \psi))
    \end{split}
\end{equation}
la seconde égalité découlant de la formule de Simpson.

Le terme $\cos(2\omega t + \phi + \psi)$ oscille autour de 0,
donc sa contribution à la moyenne est nulle.
Il reste donc le terme constant $\frac{1}{2}FG\cos(\phi - \psi)$.
Nous reconnaissons ici la forme du produit scalaire des phaseurs
$Fe^{j\phi}$ et $Ge^{j\psi}$ par rapport à la base orthonormée $(1,i)$,
divisée par deux.

Exprimons ce produit scalaire pour en avoir le cœur net:
\begin{equation}
    \begin{split}
        (Fe^{j\phi} \,|\, Ge^{j\psi})
        &= (F(\cos\phi+i\sin\phi) \,|\, G(\cos\psi+i\sin\psi)) \\
        &= FG\,\big[\cos\phi\cos\psi(1|1) + \cos\phi\sin\psi(1|i) \\
        &+ \sin\phi\cos\psi(i|1) + \sin\phi\sin\psi(i|i)\big]
    \end{split}
\end{equation}
