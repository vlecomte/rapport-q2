\chapter{Méthode des phaseurs}
\label{chap:phaseurs}

\textbf{À RÉÉCRIRE EN TANT QU'ANNEXE}

Dans de nombreux domaines de la physique,
on retrouve des phénomènes périodiques.
Une manière simplifier leur étude
est de ramèner les fonctions périodiques à une somme de sinusoïdales,
puis d'étudier ces dernières avec la méthode des phaseurs.
\begin{itemize}
    \item La première étape consiste à prendre la transformée de Fourier,
        de la fonction périodique quelconque donnée.
        Toutefois, elle demande un développement d'analyse assez important.
        Nous n'étudierons ici que des signaux d'entrée sinusoïdaux purs
        (qui correspondent d'ailleurs à des sons purs).
    \item La deuxième étape consiste à représenter chacune des sinusoïdales
        sous une forme qui simplifie les calculs classiques à effectuer,
        c'est-à-dire la résolution d'équations différentielles,
        puis plus spécifiquement la résolution de circuits électriques.
\end{itemize}

Dans le cadre de ce rapport,
nous utilisons les phaseurs pour modéliser les filtres de fréquences,
les mouvements de la membrane,
et une version simplifiée de la propagation du son.

\section{Méthode classique: filtre passe-bas}

Le filtre passe-bas consiste en une résistance $R$ et une capacité $C$
mises en série,
comme illustré dans la figure~\ref{fig:circ-passe-bas}.
Le signal d'entrée est reçu aux extrémités, et la tension de sortie est prise
aux bornes de la capacité.

\begin{figure}[h!]
    \centering
    \begin{circuitikz}
        \draw (0,0)
        node[anchor=east]{$v\ind{in}$}
        to[R=$R$, v_<=$v_R$, o-*] (3,0)
        to[C=$C$, v_<=$v_C$, -*] (6,0)
        node[ground]{}
        ;
    \end{circuitikz}
    \caption{Schéma électrique d'un circuit RC}
    \label{fig:circ-passe-bas}
\end{figure}

Définissons:
\begin{itemize}
    \item $v\ind{in}(t)$ la tension d'entrée;
    \item $v_R(t)$ la tension aux bornes de la résistance;
    \item $v_C(t)$ la tension aux bornes de la capacité;
    \item $i(t)$ le courant (de haut en bas).
\end{itemize}

À partir d'ici nous les considérerons implicitement
comme des fonctions du temps.
Nous supposons que le circuit est en régime sinusoïdal stable.

Les données dont nous disposons sont:
\begin{itemize}
    \item Loi des tensions de Kirchhoff: $v_R + v_C = v\ind{in}$
    \item Résistance: $v_R = Ri$
    \item Capacité: $i = C\,dv_C/dt$
\end{itemize}

Développons la première relation:
\begin{equation}
    \label{eq:diff-passe-bas}
    v\ind{in} = v_R + v_C = Ri + v_C = RC\,\frac{dv_C}{dt} + v_C
\end{equation}
ceci est une équation différentielle linéaire d'ordre 1 en $v_C$,
à coefficients constants, et non-homogène.
La solution de l'équation homogène correspondante est $v_C = e^{-t/RC}$.
Toutefois, il s'agit d'une exponentielle décroissante,
donc son effet n'est pas présent à l'état d'équilibre.

Posons $v\ind{in}(t) = |V\ind{in}|\cos(\omega t)$.
Puisque le signal d'entrée est sinusoïdal, on peut trouver une solution
particulière de la forme
\[
    v_C(t) = |V_C|\cos(\omega t + \phi)
    = (|V_C|\cos\phi)\,\cos(\omega t) - (|V_C|\sin\phi)\,\sin(\omega t)
\]

En substituant ces expressions dans l'équation,
puis en séparant les termes en $\cos(\omega t)$ et $\sin(\omega t)$,
on obtient:
\begin{equation}
    \left\{
        \begin{array}{ccrcr}
            |V\ind{in}| &=& |V_C|\cos\phi &-& \omega RC\ |V_C|\sin\phi \\
            0 &=& -\omega RC\ |V_C|\cos\phi &-& |V_C|\sin\phi
        \end{array}
    \right.
\end{equation}
ou encore:
\begin{equation}
    \left\{
        \begin{array}{ccl}
            |V_C|\cos\phi &=& |V\ind{in}|\,/(1+(\omega RC)^2) \\
            |V_C|\sin\phi &=& |V\ind{in}|\,(-\omega RC) / (1+(\omega RC)^2)
        \end{array}
    \right.
\end{equation}
et donc:
\begin{equation}
    \left\{
        \begin{array}{ccl}
            |V_C| &=& |V\ind{in}|\,/ \sqrt{1+(\omega RC)^2} \\
            \phi &=& \arctan(-\omega RC)
        \end{array}
    \right.
\end{equation}
où $V_C$ est l'\emph{amplitude} et $\phi$ le \emph{déphasage}
du signal de sortie.
Pour une illustration et une interprétation de ces résultats,
voir la section \ref{sec:filtres/mode} dans la partie principale du rapport.

Cette méthode a certains désavantages.
Pour commencer, comme nous pouvons le voir, les calculs à effectuer
sont assez longs.
Seule la solution particulière nous intéresse,
et il est dommage de devoir passer par une équation différentielle pour cela.

Ensuite, elle n'est pas facilement automatisable.
Ici, nous avons pris un cas très simple où l'équation différentielle
vient naturellement.
Mais dans d'autres cas,
d'une part la simple expression d'une
des lois de Kirchoff ne suffit pas,
et d'autre part il faudra parfois dériver les équations
obtenues pour faire apparaître la variable choisie.

Dès lors, une méthode plus générale et plus efficace est nécessaire.

\section{Définition de l'isomorphisme}
Commençons par étudier l'ensemble des fonctions du temps
\[
    f(t) = |F|\cos(\omega t + \phi)
\]
où $|F|$ est une constante réelle (avec éventuellement une dimension physique
\footnote{
    En réalité, le support complet des grandeurs physiques
    nécessite une extension de la définition d'espace vectoriel.
    En effet, seules des valeurs de mêmes unités sont additionnables,
    et la multiplication par un scalaire possédant des unités
    va changer les unités de la fonction.
    Dans la suite de ce raisonnement,
    nous procéderons comme si toutes les valeurs étaient sans unités.
}),
$\phi$ est une constante réelle en radians,
$t$ est une variable en secondes
et $\omega$ est un paramètre de l'ensemble, en Hertz
(c'est le même pour toutes les fonctions de l'ensemble).
Notons cet ensemble $\mathbb{S}_\omega$.

Il s'agit d'un espace vectoriel.
Nous ne détaillerons pas ici la preuve
(dont une partie se trouve dans la section~\ref{sec:math/espace-vect-sinus}),
mais en pratique cela signifie qu'une combinaison linéaire
de sinusoïdales reste une sinusoïdales,
et que ces fonctions ont un certain nombre de propriétés
typiquement associées aux vecteurs.

Revenons à notre but de représenter des sinusoïdes
par des objets mathématiques plus simples.
Les exponentielles complexes allient un caractère sinusoïdal à
la simplicité d'une exponentielle.
Elles sont donc un bon candidat potentiel.
Nous savons que:
\begin{equation}
    f(t) = |F|\cos(\omega t + \phi) = \mathfrak{Re}\{|F|e^{j(\omega t + \phi)}\}
    = \mathfrak{Re}\{|F|e^{j\phi}\cdot e^{j\omega t}\}
\end{equation}
où $j$ est l'unité imaginaire.
Remarquons que $e^{j\omega t}$ ne dépend pas de la sinusoïde représentée.
Cela signifie que la constante complexe $|F|e^{j\phi}$,
qu'on appellera \emph{phaseur}, contient
toute l'information qui caractérise la fonction $f$.
D'ailleurs, l'amplitude $|F|$ correspond module du phaseur
tandis que le déphasage $\phi$ correspond à son argument.

On peut même aller plus loin: définissons l'application $L$
qui à une fonction $f(t) = |F|\cos(\omega t + \phi)$
associe le complexe $|F|e^{j\phi}$.
On peut montrer (voir l'annexe~\ref{sec:math/isomorphisme}) qu'il s'agit d'un
isomorphisme entre $\mathbb{S}_\omega$ et $\mathbb{C}$.

Cela a deux conséquences importantes:
\begin{itemize}
    \item À toute fonction sinusoïdale correspond un et un seul phaseur
        (un isomorphisme est une bijection).
    \item On peut effectuer des sommes ou des multiplications par un scalaire
        indifféremment sur les fonctions ou sur les phaseurs correspondants.
\end{itemize}

On peut déjà entrevoir les avantages de cette représentation.
En effet,
les opérations sur les complexes sont bien plus simples et mieux connues
que celles sur les fonctions.

Nous noterons en général les fonctions en minuscules,
les phaseurs en majuscule,
et les amplitudes comme le module des phaseurs.
Par exemple: $v(t) = |V|\cos(\omega t + \phi)$ et
$V = |V|e^{j\phi}$.

Dans les sections suivantes, nous allons montrer en détail
les nombreux avantages des phaseurs.

\section{Différentiation}
\label{sec:phaseurs/diff}

Nous allons maintenant déterminer l'opération qu'il faut effectuer
sur le phaseur d'une fonction pour obtenir le phaseur de sa dérivée temporelle.

Prenons comme d'habitude une fonction $f(t) = |F|\cos(\omega t + \phi)$
de phaseur $F = |F|e^{j\phi}$
et dérivons-la:
\begin{equation}
    \frac{d}{dt}f(t) = \omega (-|F|\sin(\omega t + \phi))
    = \omega |F| \cos(\omega t + \pi/2 + \phi)
\end{equation}

Le phaseur correspondant $F'$ vaut:
\begin{equation}
    F' = \omega |F| e^{j(\pi/2 + \phi)} = \omega |F| e^{j\pi/2} e^{j\phi}
    = j\omega |F| e^{j\phi} = j\omega F
\end{equation}

On découvre donc que quand on dérive une fonction sinusoïdale,
son phaseur est simplement multiplié par $j\omega$.
\footnote{
    Alternativement, on aurait pu voir que
    $(\mathfrak{Re}\{ |F|e^{j\phi} \cdot e^{j\omega t}  \})'
    = \mathfrak{Re}\{(|F|e^{j\phi} \cdot e^{j\omega t})'\}
    = \mathfrak{Re}\{j\omega |F| e^{j\phi} \cdot e^{j\omega t}\}$,
    ce qui montre également que le phaseur $|F|e^{j\phi}$
    est multiplié par $j\omega$.
}
Cela a comme conséquence immédiate de simplifier énormément
le calcul de solutions particulières sinusoïdales d'équations différentielles,
les transformant en de simples équations linéaires sur les complexes.

Reprenons l'équation différentielle décrivant le filtre passe-bas
\eqref{eq:diff-passe-bas}:
\[
    v\ind{in} = RC\frac{dv_C}{dt} + v_C
\]
on peut la réécrire sous forme de phaseurs:
\begin{equation}
    V\ind{in} = RC(j\omega)V_C + V_C
\end{equation}
ce qui donne immédiatement la solution:
\begin{equation}
    V_C = \frac{V\ind{in}}{1 + j\omega RC}
\end{equation}

On peut ensuite extraire l'amplitude et le déphasage de $v_C$
en calculant respectivement le module et l'argument de $\overline{V_C}$:
\begin{equation}
    \left\{
        \begin{array}{ccl}
            |V_C| &=& |V\ind{in}|\,/\,|1+j\omega RC|
            = |V\ind{in}|\,/\sqrt{1 + (\omega RC)^2} \\
            \phi &=& \arg(V_C) = -\arg(1+j\omega RC)
            = - \arctan(\omega RC)
        \end{array}
    \right.
\end{equation}
ce qui correspond exactement aux résultats trouvés précédemment.

\section{Moyenne d'un produit}

Le produit de deux fonctions $f$ et $g$ dans $\mathbb{S}_\omega$
n'est pas interne.
En effet, la fréquence de $f \cdot g$ est doublée,
et une constante s'ajoute à la sinusoïde.

Toutefois, il est possible d'exprimer de manière intéressante
la moyenne de ce produit.
Un exemple d'application est la puissance moyenne absorbée par un circuit,
qui vaut la moyenne du produit $v\cdot i$ de la tension et du courant.

Posons $f(t) = |F|\cos(\omega t + \phi)$ et $g(t) = |G|\cos(\omega t + \psi)$.
Exprimons le produit de ces deux fonctions:
\begin{equation}
    \begin{split}
        (f\cdot g)(t) &= |F||G|\cos(\omega t + \phi)\cos(\omega t + \psi) \\
        &= \frac{|F||G|}{2}(\cos(2\omega t + \phi + \psi) + \cos(\phi - \psi))
    \end{split}
\end{equation}
la seconde égalité découlant de la formule de Simpson.

Le terme $\cos(2\omega t + \phi + \psi)$ oscille autour de 0,
donc sa contribution à la moyenne est nulle.
Il reste donc le terme constant $\frac{1}{2}|F||G|\cos(\phi - \psi)$.
Nous reconnaissons ici la forme du produit scalaire des phaseurs
$|F|e^{j\phi}$ et $|G|e^{j\psi}$ par rapport à la base orthonormée $\{1,j\}$,
divisée par deux.

Exprimons ce produit scalaire pour s'en assurer:
\begin{equation}
    \begin{array}{rcl}
        \scal{|F|e^{j\phi}}{|G|e^{j\psi}}
        &=& \scal{|F|(\cos\phi+j\sin\phi)}{|G|(\cos\psi+j\sin\psi)} \\
        &=& |F||G|\,\big[\cos\phi\cos\psi \scal{1}{1}
        + \cos\phi\sin\psi \scal{1}{j} \\
        && + \sin\phi\cos\psi \scal{j}{1}
        + \sin\phi\sin\psi \scal{j}{j}\big] \\
        &=& |F||G|(\cos\phi\cos\psi + \sin\phi\sin\psi) \\
        &=& |F||G|\cos(\phi-\psi)
    \end{array}
\end{equation}

Nous avons donc prouvé que la moyenne de $(f\cdot g)(t)$
vaut $\frac{1}{2}\scal{F}{G}$.
Cela permet notamment de tirer avantage de la forme classique $a+bj$
des complexes pour les phaseurs,
avec laquelle $\scal{a+bj}{c+dj}$ vaut simplement $ac + bd$.

