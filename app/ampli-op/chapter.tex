\chapter{Rôle de l'ampli op}
\label{chap:ampli-op}

\subsection*{Introduction}

Dans ce chapitre, nous allons nous interroger sur la nécessité
d'inclure un ampli-op pour séparer le réglage de volume
et les deux filtres.

Celui-ci permet de rendre parfaitement indépendants ces différents
réglages, et il simplifie grandement les calculs,
mais nous allons prouver qu'avec les valeurs des potentiomètres,
il peut être retiré et remplacé par un court-circuit
sans changement significatif.

\section{Résolution}
\begin{figure}[h!]
    \centering
    \begin{circuitikz}
        \draw
        (0,0) node[anchor=east]{$V\ind{in}$}
        to[R,l_=$R_{11}$, o-*] (0,-2)
        node[anchor=east]{$V_1$}
        to[R,l_=$R_{12}$] (0,-4)
        node[ground]{}
        (0,-2) to[C=$C_2$, -*] (3,-2)
        to[R=$R_2$] (3,-4)
        node[ground]{}
        (3,-2) to[R=$R_3$, -o] (6,-2)
        to[C=$C_3$] (6,-4)
        node[ground]{}
        ;
    \end{circuitikz}
    \caption{Schéma éléctrique du réglage de volume et des filtres sans ampli-op}
    \label{fig:sans-opamp}
\end{figure}

La figure~\ref{fig:sans-opamp}
