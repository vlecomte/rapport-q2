\chapter{Rôle de l'ampli op}
\label{chap:ampli-op}

\subsection*{Introduction}

Dans ce chapitre, nous allons nous interroger sur la nécessité
d'inclure un ampli-op pour séparer le réglage de volume
et les deux filtres.

Celui-ci permet de rendre parfaitement indépendants ces différents
réglages, et il simplifie grandement les calculs,
mais nous allons prouver qu'avec les valeurs des potentiomètres,
il peut être retiré et remplacé par un court-circuit
sans changement significatif.

\section{Résolution}
\begin{figure}[h!]
    \centering
    \begin{circuitikz}
        \draw
        (0,0) node[anchor=east]{$V\ind{in}$}
        to[R,l_=$R_{11}$, o-*] (0,-2)
        node[anchor=east]{$V_1$}
        to[R,l_=$R_{12}$] (0,-4)
        node[ground]{}
        (0,-2) to[C=$C_2$, -*] (3,-2)
        to[R=$R_2$] (3,-4)
        node[ground]{}
        (3,-2) to[R=$R_3$, -o] (6,-2)
        to[C=$C_3$] (6,-4)
        node[ground]{}
        ;
    \end{circuitikz}
    \caption{Schéma éléctrique du réglage de volume et des filtres sans ampli-op}
    \label{fig:sans-opamp}
\end{figure}

La figure~\ref{fig:sans-opamp} montre le schéma électrique sans ampli-op,
de la sortie du jack à l'entrée de l'ampli IC.
Nous pouvons résoudre ce circuit par la méthode des nœuds généralisée,
dont nous parlons dans la section~\ref{sec:phaseurs/impedance}.

Supposons $V\ind{in}$ connu, cela nous donne le système:
\begin{equation}
    \left\{
    \begin{array}{rcl}
        \frac{V\ind{in}-V_1}{R_{11}} &=&
        \frac{V_1}{R_{12}}+\frac{V_1-V_2}{1/j\omega C_2} \\
        \frac{V_1-V_2}{1/j\omega C_2} &=&
        \frac{V_2}{R_2} + \frac{V_2-V_3}{R_3} \\
        \frac{V_2-V_3}{R_3} &=& \frac{V_3}{1/j\omega C_3}
    \end{array}
    \right.
\end{equation}
Posons $R_1 = R_{11}+R_{12}$ pour les calculs.

Après une résolution assez fastidieuse, nous obtenons une fonction de transfert
totale
\begin{equation} \label{eq:Htot-complique}
    H\ind{tot}(j\omega) = \frac{j\omega C(R_2R_{12})}
    {\begin{array}{l}
            R_1 + j\omega C(R_1R_3+2R_1R_2+R_{11}R_{12})\\
        + (j\omega C)^2(R_1R_2R_3 + R_{11}R_{12}R_3 + 2R_{11}R_{12}R_2
        - R_{11}R_{12}R_2)
\end{array}}
\end{equation}
avec l'hypothèse $C_2 = C_3 = C$.

\section{Interprétation}

À titre de comparaison, la fonction de transfert totale avec l'ampli-op est
\begin{equation}
    H\ind{tot}(j\omega) = \frac{R_{12}}{R_1}\cdot
    \frac{j\omega CR_2}
    {1+j\omega CR_2 + (j\omega C)^2R_2R_3}
\end{equation}

Nous pouvons simplifier l'expression~\eqref{eq:Htot-complique}
par les suppositions, raisonnables dans la plupart des cas
pour nos choix de potentiomètres:
\begin{equation}
    R_2 \gg R_3 \gg R_1 \geq R_{11},R_{12}
\end{equation}
ce qui nous donne l'expression
\begin{equation}
    \begin{split}
        H\ind{tot}(j\omega) &= \frac{j\omega C (R_{12} R_2)}
        {R_1 + j\omega C(R_1\cdot 2R_2) + (j\omega C)^2(R_1\cdot R_2R_3)} \\
        &= \frac{R_{12}}{R_1}\cdot
        \frac{j\omega C R_2}{1+2j\omega CR_2 + (j\omega C)^2R_2R_3}
    \end{split}
\end{equation}
qui diffère de la fonction de transfert avec l'ampli-op uniquement
par le facteur 2 devant $j\omega CR_2$.

Nous pouvons même faire mieux,
si l'on règle les potentiomètres différemment.
Si l'on les met chacun sur une nouvelle position:
\begin{itemize}
    \item $R_{12}' = 2R_{12}$;
    \item $R_2' = R_2/2$;
    \item $R_3' = 2R_3$;
\end{itemize}
alors la fonction de transfert devient exactement identique:
\begin{equation}
    \begin{split}
        H\ind{tot}(j\omega) &= \frac{2R_{12}}{R_1} \cdot\frac{j\omega C R_2/2}
        {1+2j\omega CR_2/2 + (j\omega C)^2R_2/2\cdot 2R_3}\\
        &= \frac{R_{12}}{R_1}\cdot\frac{j\omega CR_2}
        {1+j\omega CR_2 + (j\omega C)^2R_2R_3}
    \end{split}
\end{equation}

Dès lors, nous avons montré que, sous nos hypothèses, le circuit sans ampli-op
donne un contrôle décalé, mais équivalent.
