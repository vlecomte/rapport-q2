\subsection{Tendances}

\textbf{\textsc{À bouger en annexe}}

Avant de calculer les gains et déphasages en détail,
nous allons mettre en évidence des tendances dans les fonctions de transfert.

Commençons par le filtre passe-haut, dont la fonction de transfert est
$H\ind{ph}(j\omega) = \frac{j\omega R_2C_2}{1+j\omega R_2C_2}$.
Le dénominateur contient deux termes qui «dominent» à tour de rôle:
le premier, 1, sera le plus grand en norme pour de petites valeurs de $\omega$,
jusqu'au point $\omega = 1/R_2C_2$, où $j\omega R_2C_2$, qui grandit
avec $\omega$, «prendra le relais».
Notons cette pulsation critique $\omega\ind{bas}= 1/R_2C_2$ et
appelons-la \emph{pulsation de coupure}.

Plus précisément,
\begin{itemize}
    \item quand $\omega << \omega\ind{bas}$,
        $H\ind{ph} \approx \frac{j\omega R_2C_2}{1} = j\omega R_2C_2$;
    \item quand $\omega >> \omega\ind{bas}$,
        $H\ind{ph} \approx \frac{j\omega R_2C_2}{j\omega R_2C_2} = 1$.
\end{itemize}

Cette rapide analyse confirme notre intuition et la précise:
\begin{itemize}
    \item pour de petites pulsations,
        le signal est de moins en moins atténué et est en avance de $\pi/2$
        sur l'entrée (à cause du facteur $j$);
    \item à partir de la pulsation de coupure le signal sera presque
        restitué à l'identique.
\end{itemize}

Pour le filtre passe-bas, la situation est analogue.
Le dénominateur fait apparaître une pulsation de coupure
$\omega\ind{haut} = 1/R_3C_3$.
À partir de la fonction de transfert
$H\ind{pb}(j\omega) = \frac{1}{1 + j\omega R_3C_3}$,
on découvre que:
\begin{itemize}
    \item quand $\omega << \omega\ind{haut}$, $H\ind{pb} \approx 1$;
    \item quand $\omega >> \omega\ind{haut}$, $H\ind{pb}
        \approx \frac{1}{j\omega R_3C_3}$;
\end{itemize}
ce qui confirme et précise notre intuition:
\begin{itemize}
    \item pour de petites pulsations, le signal sera presque restitué
        à l'identique;
    \item à partir de la pulsation de coupure, le signal est de plus en plus
        atténué
        et est en retard de $\pi/2$ sur l'entrée (à cause du facteur $1/j$).
\end{itemize}
