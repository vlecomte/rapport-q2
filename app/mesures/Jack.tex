\section{Sortie Jack 3.5 mm}

L'objectif de ces mesures était de mesurer la tension de sortie
d'un ordinateur au niveau de sa prise Jack à différents volumes
pour déterminer les précautions à prendre lors du branchement à notre circuit.

Pour étudier ce signal,
nous avons effectué des mesures sur un signal sinusoïdal
produit par le logiciel Audacity en faisant varier
son amplitude relative ainsi que le volume général
de l'ordinateur, pour une fréquence arbitraire.
Nous avons vérifié que la fréquence
n'avait aucune influence sur la tension de sortie. 

Les mesures ont été faites à une fréquence de $100\,\hertz$ et avec une
amplitude du signal variant de $0.3$ à $1.0$.
Pour chacune de ces amplitudes, nous avons réglé
volume à $100\,\%$, $75\,\%$, et $50\,\%$.

Les différentes valeurs obtenues sont reprises
dans la table~\ref{tab : Sortie Jack}.
Nous pouvons voir que le réglage de volume n'est pas linéaire.

Ces mesures ont été prises au moyen d'un multimètre connecté à la borne négative et la masse du connecteur. Il s'agit donc de valeurs RMS.

\begin{table}
\centering
\begin{tabu}{r|r|r|r}
\toprule

\textbf{Fréquence} [\hertz] &
\textbf{Amplitude} &
\textbf{Volume} [\%] &
\textbf{Tension} [\volt] \\
\hline
		100 & 1.0 & 100 & 1.361 \\
%	\cline{3-4}
	 	& & 75 & 0.511 \\
%	\cline{3-4}
	 	& & 50 & 0.161 \\
	 \cline{2-4}
		 & 0.8 & 100 & 1.090 \\
%        \cline{3-4}
		 & & 75 & 0.383 \\
%		  \cline{3-4}
		 & & 50 & 0.120 \\
	 \cline{2-4}
	 	& 0.5 & 100 & 0.681\\
%		 \cline{3-4}
	 	& & 75 & 0.255 \\
%		 \cline{3-4}
	 	& & 50 & 0.080 \\
	 \cline{2-4}
	 	& 0.3 & 100 & 0.407 \\
%		 \cline{3-4}
	 	& & 75 & 0.153 \\
%		 \cline{3-4}
	 	& & 50 & 0.048 \\
\bottomrule
\end{tabu}
\caption{Mesures de tension RMS de sortie du Jack $3.5\,\milli\meter$.}
\label{tab : Sortie Jack}
\end{table}
