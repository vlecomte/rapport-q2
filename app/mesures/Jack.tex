\section{Sortie Jack $3.5$\milli \meter}

Pour étudier le signal de sortie de notre prise Jack $3.5$ \milli \meter nous avons effectué des mesures sous un signal d'entrée sinusoïdal en faisant varier
son amplitude et le volume pour une fréquence choisie arbitrairement. La fréquence n'a en effet aucune influence sur la tension de sortie. 

Nous avons pris des mesures pour une fréquence de $100$\hertz et avons modifié l'amplitude du signal de avec des valeurs allant de $0,3$ à $1$. Pour chacune de ces amplitudes, nous avons fait varier le pourcentage du volume pour des valeurs de 100\%,75\%, et 50\%.

Les différentes valeurs obtenues sont reprises dans la table ~\ref{tab : Sortie Jack} 

Ces mesures ont été prises au moyen d'un multimètre connecté à la borne négative et la masse de la prise Jack. Il s'agit donc de valeurs RMS.




\begin{table}
\begin {center}
\begin{tabular}{|r|r|r|r|}
\hline

	Fréquence [\hertz] & Amplitude & Volume [\%] & Tension [\volt] \\
\hline
		100 & 1 & 100 & 1.361 \\
	 	& & 75 & 0.511 \\
	 	& & 50 & 0.161 \\
	 \cline{2-4}
		 & 0.8 & 100 & 1.090 \\
		 \cline{3-5}
		 & & 75 & 0.383 \\
		  \cline{3-5}
		 & & 50 & 0.120 \\
	 \cline{2-4}
	 	& 0.5 & 100 & 0.681\\
		 \cline{3-5}
	 	& & 75 & 0.255 \\
		 \cline{3-5}
	 	& & 50 & 0.080 \\
	 \cline{2-4}
	 	& 0.3 & 100 & 0.407 \\
		 \cline{3-5}
	 	& & 75 & 0.153 \\
		 \cline{3-5}
	 	& & 50 & 0.048 \\
\hline
\end{tabular}
\caption{Mesures de tension de sortie du Jack $3.5$ \milli \meter (tension RMS).}
\label{tab : Sortie Jack}
\end{center}
\end{table}
