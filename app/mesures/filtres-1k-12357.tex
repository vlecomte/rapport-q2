\section{Filtres passe-bas et passe-haut isolés}
\label{sec:mesures/filtres-1k-12357}

Ces mesures avaient pour but de confirmer la modélisation théorique
des filtres passe-bas et passe-haut.
Nous avons mis une résistance $R=1\,\kilo\ohm$ et une capacité
$C=470\,\nano\farad$ en série.
Un générateur de signaux imposait une tension sinusoïdale aux bornes.

Nous avons d'abord mesuré les valeurs de la résistance et de la capacité
individuellement au multimètre pour éliminer les incertitudes liées
à la fabrication.

Nous avons mesuré $R=1.003\,\kilo\ohm$ et $C=433\,\nano\farad$.

Ensuite, nous avons placé des voltmètres en mode AC aux bornes du générateur,
de la résistance et de la capacité.
Nous avons fait varier la fréquence ($f$) et noté à chaque point
la fréquence et les trois tensions efficaces (RMS) correspondantes
($V\ind{in}$, $V_R$ et $V_C$).
Ces mesures sont rassemblées dans la table~\ref{tab:filtres-1k}

\begin{table}
    \centering
    \begin{tabu} to .6\linewidth
        {r|X[3r]X[2r]@{.}X[2l]X[r]@{.}X[2l]X[r]@{.}X[2l]}
        \toprule
        \multicolumn{1}{c}{\No} &
        \multicolumn{1}{c}{$f$ (\hertz)} &
        \multicolumn{2}{c}{$V\ind{in}$ (\volt)} &
        \multicolumn{2}{c}{$V_R$ (\volt)} &
        \multicolumn{2}{c}{$V_C$ (\volt)} \\
        \hline
        1 & 10 & 5&75 & 0&181 & 5&78 \\
        2 & 20 & 5&75 & 0&351 & 5&75 \\
        3 & 29 & 5&75 & 0&514 & 5&72 \\
        4 & 51 & 5&74 & 0&896 & 5&65 \\
        5 & 69 & 5&73 & 1&18 & 5&57 \\
        6 & 99 & 5&73 & 1&66 & 5&45 \\
        7 & 197 & 5&67 & 2&90 & 4&79 \\
        8 & 289 & 5&62 & 3&64 & 4&18 \\
        9 & 505 & 5&55 & 4&57 & 3&03 \\
        10 & 688 & 5&52 & 4&89 & 2&42 \\
        11 & 994 & 5&52 & 5&16 & 1&80 \\
        12 & 1964 & 5&51 & 5&38 & 0&980 \\
        13 & 2940 & 5&54 & 5&46 & 0&691 \\
        14 & 5070 & 5&67 & 5&64 & 0&433 \\
        15 & 6920 & 5&86 & 5&84 & 0&341 \\
        16 & 10010 & 6&34 & 6&39 & 0&251 \\
        \bottomrule
    \end{tabu}
    \caption{Mesures de fréquences et tensions RMS d'un circuit RC}
    \label{tab:filtres-1k}
\end{table}
