\section{Choix de la méthode}

Dans cette section, nous allons justifier
le choix que nous avons effectué d'utiliser la première méthode,
en répondant à deux critiques auxquelles nous avons pu penser,%
\footnote{
    Nous ne considérerons pas ici les erreurs numériques
    dues au troncage des nombres flottants.
}
puis en présentant une version améliorée et plus générale.

Une première critique possible contre cette méthode
est qu'elle utilise un polynôme interpolateur $I$.
Or, construire un tel polynôme a pour complexité $O(n\log n)$.
\cite{lagrange-complexity}
Dès lors, on pourrait penser que cela va augmenter sa complexité totale.

Toutefois, comme expliqué dans la section précédente
(\ref{sec:approx-lin/temps}),
il n'est pas nécessaire de calculer l'expression du polynôme $I$,
et la méthode a la même complexité que les deux autres.

Une deuxième critique que l'on peut y trouver est que,
quand le nombre de points de mesure augmente,
les polynômes interpolateurs ont tendance à varier énormément
dès que l'on s'éloigne en abcisse,
à cause des termes de degré supérieur.
Il y a donc lieu de s'inquiéter par rapport à la fidélité
de l'approximation linéaire que la méthode~1 va produire.

Néanmoins, cela n'a pas d'effet sur le résultat, comme nous l'avons
prouvé en section~\ref{sec:approx-lin/resultat},
car à nouveau les seules ordonnées qui interviennent dans le calcul
sont celles des points de mesure.
