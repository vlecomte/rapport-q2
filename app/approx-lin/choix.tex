\section{Choix de la méthode}

Dans cette section, nous allons justifier
le choix que nous avons effectué d'utiliser la première méthode,
en répondant à deux critiques auxquelles nous avons pu penser,%
\footnote{
    Nous ne considérerons pas ici les erreurs numériques
    dues au troncage des nombres flottants.
}
puis en présentant une version améliorée et plus générale.

\subsection{Réponse aux critiques}

Une première critique possible contre cette méthode
est qu'elle utilise un polynôme interpolateur $I$.
Or, construire un tel polynôme a pour complexité $O(n\log n)$.
\cite{lagrange-complexity}
Dès lors, on pourrait penser que cela va augmenter sa complexité totale.

Toutefois, comme expliqué dans la section précédente
(\ref{sec:approx-lin/temps}),
il n'est pas nécessaire de calculer l'expression du polynôme $I$,
et la méthode a la même complexité que les deux autres.

Une deuxième critique que l'on peut y trouver est que,
quand le nombre de points de mesure augmente,
les polynômes interpolateurs ont tendance à varier énormément
dès que l'on s'éloigne en abcisse,
à cause des termes de degré supérieur.
Il y a donc lieu de s'inquiéter par rapport à la fidélité
de l'approximation linéaire que la méthode~1 va produire.

Néanmoins, cela n'a pas d'effet sur le résultat, comme nous l'avons
prouvé en section~\ref{sec:approx-lin/resultat},
car à nouveau les seules ordonnées qui interviennent dans le calcul
sont celles des points de mesure.

\subsection{Généralisation}

Les réponses aux deux critiques ci-dessus tendent à indiquer
que le polynôme interpolateur $I$ n'est pas un élément clé du raisonnement,
qu'il n'est pas indispensable.
Nous allons donc définir une version modifiée de la méthode~1,
qui fait directement intervenir les points de mesure.

Nous partons de la constatation que, lorsque l'on veut projeter
un élement d'un espace $E$ vers un espace $V$, ici $\reals[X]_1$,
le produit scalaire ne doit être défini positif que
pour les éléments non-nuls de $V$, et il suffit qu'il soit semi-défini
positif pour les éléments de $E$.
\cite{semi-def-pos}

Dès lors, en gardant le même produit scalaire, il n'est pas nécessaire
de se limiter aux polynômes de degré $n-1$.
En effet, pour une fonction $f$ réelle quelconque, le produit
scalaire
\[
    \scal{f}{f} = f(x_1)^2+\cdots+f(x_n)^2
\]
est bien positif ou nul.

Nous pouvons donc directement projeter sur l'espace des droites $y=kX+l$
la fonction $f:A \supset \{x_1,\ldots,x_n\} \to \reals$ de nos mesures
qui respecte
\[
    f(x_1) = y_1, \ldots, f(x_n) = y_n
\]
en utilisant une formule identique à
la~\eqref{eq:proj-ortho-I} mais cette fois pour $f$:
\begin{equation}
    D = \scal{f}{e_1}e_1 + \scal{f}{e_2}e_2
\end{equation}

Cette méthode généralisée a aussi l'avantage de supporter
des produits scalaires plus exotiques,
comme
\[
    \scal{f}{g} = \int_a^b{f(x)g(x)\,dx}
\]
dans le cas où il ne s'agit pas uniquement de mesures discrètes.
