\subsection*{Introduction}

Dans le cadre de notre de cours de maths,
il nous a été demandé d'identifier dans les mesures de gain effectuées
sur le filtre passe-bas,
des tendances linéaires qui permettent d'identifier expérimentalement
la fréquence de coupure.

Pour cela, il faut passer en double échelle logarithmique,
puis utiliser l'une des trois méthodes qui nous ont été presentées:
deux liées aux espaces euclidiens et une liée au calcul différentiel
à plusieurs variables.

Dans ce chapitre, nous allons choisir une de ces trois méthodes
par une étude de leur résultat et leur complexité algorithmique en temps,
puis l'appliquer aux mesures réalisées en laboratoire
(voir section~\ref{sec:mesures/filtres-1k-12357}).

\subsubsection*{Plan du chapitre}
\begin{enumerate}
    \item Nous commencerons par la \emph{présentation} du problème et
        des trois méthodes,
        ainsi que le sens et l'intérêt de passer en échelle log-log.
    \item Puis nous montrerons que ces méthodes produisent
        le même \emph{résultat},
        c'est-à-dire qu'ils choisissent la même droite.
    \item Ensuite, nous prouverons qu'elles ont la même \emph{complexité
        en temps}, c'est-à-dire que leurs temps d'exécutions sont comparables
        quand le nombre de points augmente.
    \item Après cela, nous effectuerons malgré tout un \emph{choix}
        entre les trois méthodes.
    \item Enfin, nous présenterons rapidement l'\emph{application}
        de cette méthode aux mesures effectuées.
\end{enumerate}
