\subsection*{Introduction}

Dans le cadre de notre de cours de maths,
il nous a été demandé d'identifier dans les mesures de gain effectuées
sur le filtre passe-bas,
des tendances linéaires qui permettent d'identifier expérimentalement
la fréquence de coupure.

Pour cela, il faut passer en double échelle logarithmique,
puis utiliser l'une des trois méthodes qui nous ont été presentées:
deux liées aux espaces euclidiens et une liée au calcul différentiel
à plusieurs variables.

Dans ce chapitre, nous allons choisir une de ces trois méthodes
par une étude de leur résultat et leur complexité algorithmique en temps,
puis l'appliquer aux mesures réalisées en laboratoire
(voir section~\ref{sec:mesures/filtres-1k-12357}).

\subsubsection*{Plan du chapitre}
