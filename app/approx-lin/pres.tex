\section{Présentation}
\label{sec:approx-lin/pres}

Dans cette section nous allons d'abord expliquer ce qu'est
«passer en double échelle logarithmique» et ensuite presenter
chacune des méthodes proposées

\subsection{Double échelle logarithmique}
Pour commencer, nous avons déterminé à plusieurs fréquences
le gain en tension d'un filtre passe-bas
(sur base des mesures en section~\ref{sec:mesures/filtres-1k-12357}),
ce qui constitue nos points de mesure.

Ensuite, nous les passons en double-échelle logarithmique (ou \emph{log-log})
ce qui signifie en pratique qu'au lieu de représenter le point $(x,y)$,
nous représenterons sur le graphe le point $(x',y') = (\log x,\log y)$.%
\footnote{
    Dans ce chapitre et le suivant, nous utiliserons
    $\log$ pour noter la fonction $x \to \log_{10} x$
    et $\exp$ pour noter la fonction $x \to 10^x$.
}
La figure~\ref{fig:regular-to-loglog} montre les graphes
avant et après ce changement de variable.

Outre la très nette amélioration de visibilité,
l'échelle log-log permet de faire apparaître clairement
des tendances polynômiales.
En effet, si les points d'origine obéissent à la relation $y = f(x)$,
après le changement de variable, celle-ci devient:
\begin{equation}
    \exp(y') = f(\exp(x')) \Longrightarrow y' = \log(f(\exp(x'))
\end{equation}
transformant la fonction $f$ en une fonction $f' = \log\circ f\circ\exp$.
Dès lors, si $f(x) = ax^k$, la courbe dessinée aura pour
équation
\begin{equation}
    y' = f'(x') = \log(a(\exp(x')^k)) = kx'+\log a
\end{equation}
soit une droite.

Dans notre cas, le gain théorique a pour équation
$G\ind{pb}(\omega) = \frac{1}{\sqrt{1+(RC\omega)^2}}$.
Nous pouvons étudier son comportement en deux parties:
\begin{itemize}
    \item si $\omega << 1/RC$, alors $1 >> (RC\omega)^2$,
        donc $G\ind{pb} \approx 1$;
    \item si $\omega >> 1/RC$, alors $1 << (RC\omega)^2$,
        donc $G\ind{pb} \approx 1/(RC\omega)$.
\end{itemize}

Il s'agit bien là de deux d'équation de la forme $f(x) = ax^k$,
ce qui entraîne donc l'apparition des droites dans le graphe.
Plus précisément, on peut y voir
une première partie s'approchant d'une droite horizontale et
une seconde partie s'approchant d'une droite décroissante,
de pente $-1$.

Notre objectif est de déterminer de façon précise
l'intersection de ces deux droites
pour calculer la pulsation de coupure $\omega_c$,
ou fréquence de coupure $f_c$,
à laquelle le changement de comportement s'effectue.
