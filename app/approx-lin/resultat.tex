\section{Équivalence des résultats}
\label{sec:approx-lin/resultat}

Dans cette section nous allons prouver que les méthodes~1 et~3
reviennent à minimiser la fonction $\dist(D)$ de la méthode~2,
et que cette fonction admet un minimum global unique.
Nous en conclurons que les trois méthodes ont toujours un résultat,
et qu'il est toujours identique.

Commençons par la méthode~1.
Nous savons que la projection orthogonale d'un vecteur $\vec{v}$
sur un sous-espace $V$ est l'unique élément $P_V(\vec{v})$
de $V$ qui minimise la distance
\begin{equation}
    \dist(P_V(\vec{v}),\vec{v}) =
    \sqrt{\scal{P_V(\vec{v})-\vec{v}}{P_V(\vec{v})-\vec{v}}}
\end{equation}
Dans notre cas, on sait donc que la droite $D$ minimise
le produit scalaire
\begin{equation}
    \begin{aligned}
        \sqrt{\scal{D-I}{D-I}}
        &= \sqrt{(D(x_1)-I(x_1))^2+\cdots+(D(x_n)-I(x_n))^2} \\
        &= \sqrt{(D(x_1)-y_1)^2+\cdots+(D(x_n)-y_n)^2}
    \end{aligned}
\end{equation}
la deuxième égalité découle de la définition du polynôme $I$.
Ceci est exactement l'expression de la fonction $\dist(D)$ recherchée.
De plus, l'ensemble de destination de la projection $\reals[X]_1$
est finiment engendré, ce qui garantit l'existence et donc l'unicité
du minimum.

Il ne nous reste plus qu'à ramener la méthode~3 à la méthode~2.
La projection effectuée sur l'espace colonne revient à trouver
les coefficients $\alpha,\beta$ de la combinaison linéaire
\[
    \alpha
    \left(
        \begin{array}{c}
            x_1\\\vdots\\x_n
        \end{array}
    \right)
    +\beta
    \left(
        \begin{array}{c}
            1\\\vdots\\1
        \end{array}
    \right)
    \mbox{ la plus proche de }
    \left(
        \begin{array}{c}
            y_1\\\vdots\\y_n
        \end{array}
    \right)
\]
donc celle qui minimise la distance (basée sur le produit scalaire canonique)
\[
    \sqrt{(\alpha x_1 + \beta - y_1)^2+\cdots+(\alpha x_n + \beta - y_n)^2}
\]

Or, une fois cette projection réinjectée dans le système $S$,
il est clair que la solution $(k,l)$ est égale à $(\alpha,\beta)$.
Dès lors, la méthode~3 trouve également des $k,l$ minimisant
la distance
\[
    \sqrt{(kx_1+l-y_1)^2+\cdots+(kx_n+l-y_n)^2}
\]
ou encore
\[
    \sqrt{(D(x_1)-y_1)^2+\cdots+(D(x_n)-y_n)^2}
\]
ce qui conclut notre preuve.\qed
