\section{Application aux mesures}

Revenons maintenant au but premier de ce chapitre,
c'est-à-dire déterminer la fréquence de coupure d'un filtre passe-bas.
Pour faire cela, nous allons utiliser les mesures décrites en
section~\ref{sec:mesures/filtres-1k}.

Afin de déterminer les deux droites, nous utilisons les points de mesure
\no 1--5 à gauche et \no 12--16 à droite.
Les droites trouvées ont pour équation%
\footnote{
    Notez bien que les logarithmes sont en base $e$ dans l'implémentation.
}
\begin{equation}
    \left\{
    \begin{array}{rcl}
        y &=& -0.016678x+0.047312\\
        y &=& -0.91454x+5.21947
    \end{array}
    \right.
\end{equation}
ce qui s'approche raisonnablement des équations théoriques basées sur 
les valeurs mesurées de la résistance et la capacité:
\begin{equation}
    \left\{
    \begin{array}{rcl}
        y &=& 0x+0\\
        y &=& -x+5.8219
    \end{array}
    \right.
\end{equation}

La valeur de fréquence de coupure calculée
en prenant l'intersection des deux droites est $f_c=317.51\,\hertz$.

Pour une confrontation de ces résultats avec le modèle théorique,
voir la section~\ref{sec:filtres/confront}.
