\subsection{Première méthode: projection orthogonale}

À partir de maintenant supposons que nous sommes passés en échelle log-log,
et que nous avons choisi une série de $n\geq 2$ points%
\footnote{
    Les consignes suggèrent de ne considérer que 3 points pour commencer.
    Toutefois, il nous a semblé plus simple d'introduire les méthodes
    directement dans leur forme à $n$ points.
}:
\[
    (x_1,y_1),(x_2,y_2),\ldots,(x_n,y_n)
\]
pour déterminer l'une des droites qui est apparue dans le graphe.
Il s'agit donc de trouver la droit qui s'approche le plus de ses points.

La première méthode consiste à prendre le polynôme interpolateur
$I \in \mathbb{R}[X]_{n-1}$ passant par chacun des $n$ points,%
\footnote{
    Ce polynôme, appelé polynôme de Lagrange, existe et est unique.%
    \cite{lagrange-poly}
}
puis calculer sa projection orthogonale $D$
sur l'espace des droites $\mathbb{R}[X]_1$
selon le produit scalaire:
\begin{equation}
    \scal{P}{Q} = P(x_1)Q(x_1) + P(x_2)Q(x_2) + \cdots + P(x_n)Q(x_n)
\end{equation}

Pour déterminer cette projection, il faut considérer la base usuelle
$u=\{1,X\}$ de $\mathbb{R}[X]_1$
puis en extraire une base orthonormée $e=\{e_1,e_2\}$ par la méthode
de Gram-Schmidt, et ensuite calculer la droite $D$
selon la formule:
\begin{equation}
    \label{eq:proj-ortho-I}
    D = \scal{I}{e_1}e_1 + \scal{I}{e_2}e_2
\end{equation}

La symétrie et la bilinéarité du produit scalaire sont évidentes,
et nous prouvons son caractère défini positif dans
la section~\ref{sec:math/defini-pos}.
