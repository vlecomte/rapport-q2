\subsection{Troisième méthode: système surdéterminé}

La troisième méthode consiste à exprimer
le fait que la droite $D$ passe par les $n$ points.
Cela donne un système:
\begin{equation}
    S\equiv \left\{
        \begin{array}{ccc}
            kx_1+l &=& y_1 \\
            \vdots && \vdots \\
            kx_n+l &=& y_n \\
        \end{array}
    \right.
\end{equation}
qui est surdéterminé quand $n>2$.

Pour le résoudre, il faut projeter le vecteur des termes indépendants
\[
    \left(
        \begin{array}{c}
            y_1\\\vdots\\y_n
        \end{array}
    \right)
\]
sur l'espace colonne de la matrice les coefficients de $S$
\[
    \left(
        \begin{array}{cc}
            x_1&1\\
            \vdots&\vdots\\
            x_n&1
        \end{array}
    \right)
\]
selon le produit scalaire canonique de $\reals^3$.
Ensuite, il suffit de le résoudre de la façon normale.
