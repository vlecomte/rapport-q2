\subsection{Deuxième méthode: optimisation}

La deuxième méthode suit un peu plus littéralement
le problème à résoudre.
Il s'agit ici de trouver la droite $D(X) = kX+l$
qui minimise la fonction de distance $\dist(D)$
aux points de mesures, définie comme:
\begin{equation}
    \dist(D) = \sqrt{(D(x_1)-y_1)^2+\cdots+(D(x_n)-y_n)^2}
\end{equation}

La fonction $f(x)=x^2$ étant strictiment croissante sur $\mathbb{R}_{\geq 0}$,
cela revient à minimiser la fonction
\begin{equation}
    \begin{aligned}
        \dist^2(D) &= (D(x_1)-y_1)^2+\cdots+(D(x_n)-y_n)^2 \\
                   &= (kx_1+l-y_1)^2+\cdots+(kx_n+l-y_n)^2
    \end{aligned}
\end{equation}

Ceci est en réalité une fonction à deux variables $f(k,l)$,
qui, une fois l'addition effectuée, prend la forme
d'un polynôme du second degré:
\begin{equation}
    f(k,l) = Ak^2+Bl^2+Ck+Dl+Ekl+F
\end{equation}
pour des constantes $A,B,C,D,E,F \in \reals$.
Elle est définie et dérivable sur $\reals^2$,
donc le minimum ne peut se trouver qu'à un point critique.

Il ne reste qu'à calculer le gradient:
\begin{equation}
    \begin{aligned}
        \nabla f(k,l) &= \left(\frac{\partial f}{\partial x_1}(k,l),\ 
            \frac{\partial f}{\partial x_2}(k,l)\right) \\
        &= (2Ak+El+C,\ 2Bl+Ek+D)
    \end{aligned}
\end{equation}
et l'égaler à zéro pour trouver l'unique point $(k,l)$ idéal.
On peut vérifier avec la matrice hessienne qu'il s'agit bien
d'un minimum.
