\subsection{Double échelle logarithmique}
\label{subsec:approx-lin/pres/loglog}

Supposons que nous disposons d'une série points de mesures,
dont l'abcisse et l'ordonnée sont strictement positives.

Nous pouvons alors les passer
en double-échelle logarithmique (ou \emph{log-log})
ce qui signifie en pratique qu'au lieu de représenter le point $(x,y)$,
nous représenterons sur le graphe le point $(x',y') = (\log x,\log y)$.%
\footnote{
    Dans ce chapitre et le suivant, nous utiliserons
    $\log$ pour noter la fonction $x \mapsto \log_{10} x$
    et $\exp$ pour noter la fonction $x \mapsto 10^x$.
}
La figure~\ref{fig:regular-to-loglog} montre les graphes
des mesures de gain du filtre passe-bas en fonction de la fréquence
avant et après ce changement de variable.

\begin{figure}[h!]
    \centering
    \begin{tikzpicture}
        \begin{axis}[
                xlabel={Fréquence (\hertz)\qquad\qquad\qquad},
                ylabel={Gain}
            ]
            \addplot[only marks,mark=o,mark size=.5mm] table {code/gpb.dat};
        \end{axis}
    \end{tikzpicture}
    \qquad
    \begin{tikzpicture}
        \begin{loglogaxis}[
                xlabel={Fréquence (\hertz)},
                ylabel={Gain}
            ]
            \addplot[only marks,mark=o,mark size=.5mm] table {code/gpb.dat};
        \end{loglogaxis}
    \end{tikzpicture}
    \caption{Mesures de gain en fonction de la fréquence, en échelle normale et en échelle log-log}
    \label{fig:regular-to-loglog}
\end{figure}

Outre la très nette amélioration de visibilité,
l'échelle log-log permet de faire apparaître clairement
des tendances polynômiales.
En effet, si les points d'origine obéissent à la relation $y = f(x)$,
après le changement de variable, celle-ci devient:
\begin{equation}
    \exp(y') = f(\exp(x')) \Longrightarrow y' = \log(f(\exp(x'))
\end{equation}
transformant la fonction $f$ en une fonction $f' = \log\circ f\circ\exp$.
Dès lors, si $f(x) = ax^k$ ($a \in \mathbb{R}_{>0}$ et $k \in \mathbb{R}$),
la courbe dessinée aura pour équation:
\begin{equation}
    y' = f'(x') = \log(a(\exp(x')^k)) = kx'+\log a
\end{equation}
soit une droite.

Dans notre cas, le gain théorique a pour équation
$G\ind{pb}(\omega) = \frac{1}{\sqrt{1+(RC\omega)^2}}$.
Nous pouvons étudier son comportement en deux parties:
\begin{itemize}
    \item si $\omega \ll 1/RC$, alors $1 \gg (RC\omega)^2$,
        donc $G\ind{pb} \approx 1$;
    \item si $\omega \gg 1/RC$, alors $1 \ll (RC\omega)^2$,
        donc $G\ind{pb} \approx 1/(RC\omega)$.
\end{itemize}

Il s'agit bien là de deux d'équations de la forme $f(x) = ax^k$,
ce qui entraîne donc l'apparition de tendances linéaires dans le graphe.
Plus précisément, on peut y voir
une première partie s'approchant d'une droite horizontale et
une seconde partie s'approchant d'une droite décroissante
de pente $-1$.

Notre objectif est de déterminer de façon précise
l'intersection de ces deux droites
pour calculer la pulsation de coupure $\omega_c=1/RC$,
ou fréquence de coupure $f_c=1/(2\pi RC)$,
à laquelle le changement de comportement s'effectue.
