\section{Complexité en temps}
\label{sec:approx-lin/temps}

Maintenant que nous sommes sûrs que les trois méthodes donnent le même
résultat, nous pouvons nous demander quelle méthode donne ce résultat
en premier et plus particulièrement quand le nombre de mesures $n$ devient
grand.

Pour cela nous allons utiliser le concept de complexité algorithmique
en temps, et la notation grand $O$.
Une action en $O(1)$ s'exécute en \emph{temps constant} par rapport à $n$,
et une action en $O(n)$ s'exécute en \emph{temps linéaire},
directement proportionnel à $n$.

\subsubsection*{Méthode 1}

Notons qu'il n'y a pas besoin de calculer le polynôme $I$ directement.
En effet, le produit scalaire défini n'implique que ses évaluations
aux points $x_i$.
Par conséquent, les étapes de l'algorithme sont:
\begin{enumerate}
    \item Calculer une base orthonormée $\{e_1,e_2\}$ à partir de la base
        $\{1,X\}$. Cela demande un nombre constant de produits scalaires
        et de multiplications par des scalaires,
        les premiers en $O(n)$, et les dernières en $O(1)$.
        La complexité de cette étape est donc $O(n)$.
    \item Calculer la projection avec la formule
        $D=\scal{I}{e_1}e_1+\scal{I}{e_2}e_2$.
        De nouveau, il s'agit d'un nombre constant de produits scalaires
        et de multiplications par des scalaires,
        donc la complexité de cette étape est $O(n)$.
\end{enumerate}

En résumé la complexité temporelle de cette méthode est $O(n)$.

\subsubsection*{Méthode 2}

Les étapes de l'algorithme sont:
\begin{enumerate}
    \item Exprimer et additionner les termes de la fonction de distance
        $\dist^2$ pour arriver à la forme $Ak^2+Bl^2+Ck+Dl+Ekl+F$.
        Il y a $n$ termes de taille constante, donc cette étape
        est en $O(n)$
    \item Une fois arrivé à cette forme, égaler $\nabla f(k,l)$ à 0
        pour en trouver le minimum.
        Cela ne dépendant plus du tout de $n$, la complexité de cette étape
        est $O(1)$.
\end{enumerate}

En résumé, la complexité temporelle de cette méthode est $O(n)$.

\subsubsection*{Méthode 3}

Pour sa première partie, cette méthode est très similaire à la méthode~1.
Les étapes de l'algorithme sont:
\begin{enumerate}
    \item Calculer une base orthonormée $\{e_1,e_2\}$ à partir de la base
        de l'espace colonne. Cela demande un nombre constant de produits
        scalaires et de multiplications par des scalaires,
        à chaque fois en $O(n)$ (car les vecteurs sont de dimension $n$).
        La complexité de cette étape est donc $O(n)$.
    \item Calculer la projection avec la formule
        \begin{equation}
            \left(\begin{array}{c}y'_1\\\vdots\\y'_n\end{array}\right)
            =\scal{\begin{array}{c}y_1\\\vdots\\y_n\end{array}}{e_1}e_1
            +\scal{\begin{array}{c}y_1\\\vdots\\y_n\end{array}}{e_2}e_2
        \end{equation}
        De nouveau, il s'agit d'un nombre constant de produits scalaires
        et de multiplications par un scalaire, donc la complexité
        de cette étape est $O(n)$.
    \item Résoudre le système transformé
        \begin{equation}
            S' \equiv
            \left\{
                \begin{array}{ccc}
                    kx_1+l &=& y'_1\\
                    \vdots && \vdots\\
                    kx_n+l &=& y'_n
                \end{array}
            \right.
        \end{equation}
        c'est-à-dire échelonner la matrice complète du système,
        en ne gardant qu'un élément non-nul les colonnes des coefficients.
        Il y a deux colonnes, et pour chacune cela prend
        $O(n)$ opérations.
        Cette étape s'exécute donc en $O(n)$.
\end{enumerate}

Encore une fois, la complexité temporelle totale de cette méthode est $O(n)$.

\subsubsection*{Comparaison}

Les complexités en temps de ces trois méthodes sont toutes $O(n)$,
et à cause du temps pris pour utiliser la mémoire,
leurs complexités en espace sont aussi $O(n)$.%
\footnote{
    Nous savons que celles-ci sont au moins $O(n)$ à cause de l'espace
    initial occupé par les points de mesure.
}

Par conséquent, le temps et la quantité de mémoire pris par
chacune des méthodes seront comparables,
et dépendront fortement de l'implémentation.
Dès lors, il nous faudra recourir à des arguments plus subjectifs
pour faire un choix.

Des implémentations Matlab de chacune des trois méthodes sont disponibles
à la section~\ref{sec:codes/methodes}.
