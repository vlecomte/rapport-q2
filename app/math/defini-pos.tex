\section{Caractère défini positif du produit scalaire}
\label{sec:math/defini-pos}

Soient $x_1,\ldots,x_n$ des réels distincts,
et $P,Q$ des polynômes de $\reals[X]_{n-1}$.
Prouvons par induction que le produit scalaire:
\begin{equation}
    \scal{P}{Q} = P(x_1)Q(x_1) + \cdots + P(x_n)Q(x_n)
\end{equation}
est défini positif,
ou par contraposition que $\scal{P}{P}=0$ entraîne $P=0$.

\paragraph{Cas de base}
Pour $n=1$, le polynôme $P$ est constant.
Par conséquent, $\scal{P}{P} = P(x_1)^2 = 0$
implique que $P$ est nul en tout point.


\paragraph{Cas récurrent}
Supposons que:
\begin{equation}
    \scal{P}{P} = P(x_1)^2 + \cdots + P(x_n)^2 = 0
\end{equation}
Ceci étant une somme de carrés, tous ses termes sont nuls,
donc $P(x_i) = 0$ pour $i=1,\ldots,n$.

En particulier, $P$ est donc un multiple de $(X-x_n)$,
et peut s'exprimer sous la forme
\begin{equation}
    P(X) = (X-x_n)\cdot Q(X)
\end{equation}
avec $Q$ dans $\reals[X]_{n-2}$.

Puisque $(X-x_n)$ ne s'annule qu'en $x_n$,
$Q(x)$ doit s'annuler en $x_1,\ldots,x_{n-1}$,
donc
\begin{equation}
    Q(x_1)^2 + \cdots + Q(x_{n-1})^2 = 0 = \scal{Q}{Q}
\end{equation}

Par l'hypothèse de récurrence, $Q=0$, donc $P=0$. \qed
