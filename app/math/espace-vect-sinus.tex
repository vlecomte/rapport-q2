\section{Espace vectoriel des sinusoïdes}
\label{sec:math/espace-vect-sinus}

Pour des fonctions sinusoïdales $f$ et $g$, on a:
\[
    \begin{array}{rcl}
        (f+g)(t) &=& A\cos(\omega t + \phi) + B\cos(\omega t + \psi) \\
                 &=& (A\cos\phi+B\cos\psi)\cos(\omega t)
                     - (A\sin\phi+B\sin\psi)\sin(\omega t)
    \end{array}
\]

Calculons l'amplitude résultante:
\[
    \begin{array}{rcl}
        C &=& \sqrt{(A\cos\phi+B\cos\psi)^2+(A\sin\phi+B\sin\psi)^2} \\
          &=& \sqrt{A^2+B^2+2AB\cos(\phi-\psi)}
    \end{array}
\]
qu'on peut également trouver grâce à la loi des cosinus.
On peut donc réécrire la somme:
\[
    (f+g)(t) = C\left[ \frac{A\cos\phi+B\cos\psi}{C}\, \cos(\omega t)
                     - \frac{A\sin\phi+B\sin\psi}{C}\, \sin(\omega t) \right]
\]

Enfin, puisque $\left( \frac{A\cos\phi+B\cos\psi}{C} \right)^2
+ \left( \frac{A\sin\phi+B\sin\psi}{C} \right)^2 = 1$,
il existe un réel $\theta$ tel que:
\[
    \left\{
    \begin{array}{rcl}
        \cos\theta &=& \frac{A\cos\phi+B\cos\psi}{C} \\
        \sin\theta &=& \frac{A\sin\phi+B\sin\psi}{C}
    \end{array}
    \right.
\]
et donc:
\[
    \begin{array}{rcl}
        (f+g)(t) &=& C ( \cos\theta\cos(\omega t)
                       - \sin\theta\sin(\omega t) ) \\
                 &=& C\cos(\omega t + \theta)
    \end{array}
\]

\textbf{À COMPLÉTER ET RÉINTÉGRER AVEC LES PHASEURS}
