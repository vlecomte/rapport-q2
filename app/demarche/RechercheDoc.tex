
\section{Procédure de la recherche documentaire} 
\label{sec:app/demarche}

Tout au long de notre projet, nous avons été amenés à nous documenter sur diverses thématiques nécessaires pour la réalisation de notre haut-parleur et la rédaction d'un rapport complet au sujet de celui-ci.
Ces recherches nous ont été indispensables pour remplir deux objectifs : premièrement, la fabrication et l'étude du fameux haut-parleur, ce qui nécessitait de maîtriser de nouvelles notions, principalement de physique; et deuxièmement, la rédaction d'un petit \og article \fg\ sur chacun des thèmes que nous avions choisi d'approfondir, à savoir les membranes et les filtres passe-bande. (voir \textbf{section XX REF}). %mettre la REFERENCE
\newline

Pour commencer, comme nous en savions très peu sur les haut-parleurs, nous avons lu un maximum d’informations très générales sur le sujet, principalement sur des sites internet et dans des encyclopédies. Afin de trouver des articles, nous recherchions les mots-clefs \og haut-parleur \fg\ et \og loudspeaker \fg\ dans les moteurs de recherche et dans l'index des encyclopédies. 
Cette étape \og d'éclaireur \fg\ nous a permis d'avoir un bonne vue d’ensemble sur la composition et le fonctionnement d’un haut-parleur, mais également d'identifier les concepts de physique qui jouent un rôle important dans un haut-parleur.  


Ensuite, il nous a fallu étudier de plus près les concepts nouveaux que nous avions repéré lors de notre première recherche : l'électromagnétisme, l'électrodynamique, l’acoustique, les filtres passe-bande, les amplificateurs,... Comme les sources internet devenaient de moins en moins fiables sur ces sujets plus \og théoriques \fg\ et de physique plus poussée, nous nous sommes tournés vers des ouvrages plus sérieux, tels que des encyclopédies et des livres, ainsi que notre livre de référence de physique \textbf{REF} %mettre la REFERENCE  
qui nous a particulièrement aidé. Une fois la théorie comprise, nous pouvions la mettre en pratique lors du dimensionnement de notre électro-aimant et en analysant les propriétés de notre circuit électrique. 
\newline

À ce stade-ci, nous avions acquis une compréhension suffisante des différentes parties du haut-parleur et des thématiques qui y sont liées que pour pouvoir choisir de manière pertinente deux thèmes à développer dans notre rapport. Notre premier choix s'est  porté sur les membranes (leurs formes, dimensions, matériaux, propriétés etc.) car ces informations nous serviraient explicitement pour la confection matérielle du haut-parleur. Le second thème que nous avons choisi est le filtre passe-bande, notamment car nous avions repéré dans un livre une méthode de résolution de circuits à filtre (R-L-C) innovante et concise que nous trouvions intéressante à exposer et à comparer avec celle vue en cours de physique. 

A partir de ces deux thèmes nous avons rédigé une liste de mots-clés qui s'y rapportaient (que vous trouverez dans la figure \textbf{REF}).%mettre la REFERENCE schéma en araignée
 Nous nous en sommes servis pour rechercher dans l'index de livres, dans les moteurs de recherches pour trouver des brevets et sites fiables. Cette liste était régulièrement modifiée au cours de nos lectures. Pour la membrane, nous avons principalement récolté des brevets et des sites internet, tandis que pour les filtres, il s'agissait surtout de livres. 
Enfin, nous avons sélectionné parmi les nombreuses sources trouvées celles qui contenaient les informations nécessaires pour pouvoir rédiger un article sur les membranes et sur les filtres.
\newline

Ce travail nous a permis de nous rendre compte à quelle point la recherche documentaire est omniprésente dans tout travail scientifique pour acquérir les bases théoriques.
\newline

Vous trouverez dans les pages qui suivent une liste de ces ressources bibliographiques, ainsi que des traces d'extraits qui nous ont été utiles.

 
 
 
 
 


