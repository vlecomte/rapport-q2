\chapter{Conclusion}

\section{Bilan des réalisations}

Il est maintenant temps de faire le bilan du fonctionnement et des réalisations concernant notre haut-parleur, ainsi que de voir si celui-ci correspond aux spécifications de notre cahier des charges.

Point de vue réalisation techniques, le système final est constitué d'une prise Jack TRS 3,5 mm permettant le transfert d'un signal provenant dans baladeur de musique ou d'un smartphone vers un circuit électrique. Ce circuit est équipé d'un potentiomètre permettant le réglage en volume, d'un filtre passe-haut et d'un filtre basse-bas permettants le réglage en basses en en aigus. La champ magnétique est produit par un électro-aimant, alimenté par une source de tension de laboratoire. Le son, lui, est produit grâce à une membrane en papier accrochée à une bobine mobile, et reliée au circuit. Il est donc conforme au attentes du Cahier des Charges.

En plus, des spécifications du CdC, nous avons décidé, afin d'améliorer le rendu du son, de concevoir 2 hauts-parleurs différents. Un petit assurant la fonction de tweeter et un plus grand comme woofer. Le premier favorisant donc les aigus et le second les basses.

Cependant, certains points auraient pu être amélioré. En effet, même poussé au maximum, le volume du son reste assez faible.  Ce point aurait pu être amélioré en augmentant le nombre de spires de la bobine fixe / électroaimant et donc en augmentant le champ magnétique $B$.
Un autre point améliorable est la fidélité du son. Le baffle a tendance à grésiller rapidement, cela pourrait être amélioré, entre autres, grâce à une membrane de meilleure qualité.

En ce qui concerne les échéances (voir Planning, annexe~\ref{annexe:planning}), celles-ci ont presque toujours été respectées. La partie sur les réalisations techniques a été faite avant et pendant les vacances tandis que la rédaction du rapport a plutôt été faite pendant et après les vacances.

\section{Bilan du travail en groupe}

Tout au long du projet, nous avons été amenés à travailler en groupe. Cette tâche bien que fondamentale peut parfois s'avérer difficile. Dès lors, afin de la rendre le plus agréable possible, nous avons utiliser plusieurs outils.

Dans un premier temps, nous avons établi les grandes règles à respecter dans notre groupe grâce à la confection d'une convention de groupe. Nous avons également mis en place des moyens de communication comme la création d'un groupe sur un réseau social. 

Ensuite, nous avons créé un dossier partagé sur la plateforme Google Drive. Celui-ci avait 2 objectifs principaux :
\begin{itemize}
\item La centralisation de tous nos documents et travaux.
\item La possibilité de travailler en ligne sur un même document en temps réel.
\end{itemize}

Enfin, étants désireux d'écrire le rapport final avec le traitement de texte professionnel \LaTeX, nous avons appris à utiliser GitHub. Ce service nous permettant de modifier simultanément le rapport et de pouvoir l'éditer à tout moment. Ce mécanisme avait en plus comme avantage d'accepter le travail hors connexion à l'opposé de Google Drive.