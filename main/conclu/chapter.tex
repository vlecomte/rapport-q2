\chapter{Conclusion}

\section{Bilan des réalisations}

Il est maintenant temps de faire le bilan du fonctionnement et des réalisations concernant notre haut-parleur, ainsi que de voir si celui-ci correspond aux spécifications de notre cahier des charges.

Point de vue réalisation techniques, le système final est constitué d'une prise Jack TRS 3,5 mm permettant le transfert d'un signal provenant d'un baladeur de musique ou d'un smartphone vers un circuit électrique. Ce circuit est équipé d'un potentiomètre permettant le réglage en volume, d'un filtre passe-haut et d'un filtre basse-bas permettant le réglage en basses en en aigus. Le champ magnétique est produit par un électro-aimant, alimenté par une source de tension de laboratoire. Le son, lui, est produit grâce à une membrane en papier fixée à une bobine mobile, et reliée au circuit. Un amplificateur intégré au circuit permet d'amplifier 50 fois le signal reçu par le smartphone. Notre haut-parleur est donc conforme aux attentes du Cahier des Charges.

Outre les spécifications du CdC, nous avons décidé, afin d'améliorer le rendu du son, de concevoir deux hauts-parleurs différents. Un petit assurant la fonction de tweeter et un plus grand comme woofer. Le premier, favorisant donc les aigus, est un haut-parleur de type ouvert avec une membrane conique de petit diamètre. Le second, favorisant les basses est un haut-parleur de type bass-reflex. Il s'agit d'un caisson fermé avec une ouverture sur le bas afin de pouvoir récupérer une partie des ondes se trouvant à l'arrière.Sa membrane, conique également possède un diamètre plus large. Le volume du son de ces deux baffles est comparable à celui d'un smartphone à mi puissance.

Cependant, certains points auraient pu être amélioré. En effet, même poussé au maximum, le volume du son reste assez faible. Ce point aurait pu être amélioré en augmentant le nombre de spires de la bobine fixe / électro-aimant et donc en augmentant le champ magnétique $B$. 
Un autre point améliorable est la fidélité du son. Le baffle a tendance à grésiller rapidement, cela pourrait être amélioré, entre autres, grâce à une membrane de meilleure qualité et à une meilleure finition au niveau de l'assemblage.

\section{Bilan du travail en groupe}

Tout au long du projet, nous avons été amenés à travailler en groupe. Cette tâche bien que fondamentale peut parfois s'avérer difficile. Dès lors, afin de la rendre le plus agréable possible, nous avons utilisé plusieurs outils.

Dans un premier temps, nous avons établi les grandes règles à respecter dans notre groupe grâce à la confection d'une convention de groupe. Nous avons également mis en place des moyens de communication comme la création d'un groupe sur un réseau social.

Ensuite, nous avons créé un dossier partagé sur la plateforme Google Drive. Celui-ci avait deux objectifs principaux :
\begin{itemize}
\item La centralisation de tous nos documents et travaux.
\item La possibilité de travailler en ligne sur un même document en temps réel.
\end{itemize}

Enfin, étant désireux d'écrire le rapport final avec le traitement de texte professionnel \LaTeX, nous avons appris à utiliser GitHub. Ce service nous permettant de modifier simultanément le rapport et de pouvoir l'éditer à tout moment. Ce mécanisme avait en plus comme avantage de permettre le travail hors connexion à l'opposé de Google Drive.

En ce qui concerne nos points faibles et nos points forts, un de nos avantages est la gestion des échéances (voir Planning, annexe~\ref{annexe:planning}), celles-ci ont presque toujours été respectées. 
Par rapport la recherche documentaire, notre méthode de recherche aurait pu être encore plus rigoureusement et augmenter le nombre de nos sources documentaires aurait été un avantage. C'est d'ailleurs, dans ce point là que les échéances que nous nous étions fixées ont été un peu moins respectées. Le gros du travail à été effectué entre la semaine 6 et la semaine 8 (voir Planning, annexe~\ref{annexe:planning}). Cependant nous aurions pu nous y prendre plus tôt et nous avons perdu un peu de temps car notre méthode de recherche à la bibliothèque n'était pas des plus rigoureuses et des plus efficaces.


