\section{Caissons}

Il existe sur le marché, des haut-parleurs avec
différents types de caisson, tels que
les caissons basse-reflex, les caissons clos, ouverts ou encore bien d’autres.

Dans le choix du caisson de notre haut-parleur,
nous avons opté pour un caisson basse-reflex pour les graves
et un caisson ouvert pour les aigus.
Le caisson bass-reflex est entièrement fermé, à l'exception d'une petite ouverture circulaire  à l'avant qui permet de récupérer les ondes sonores émises à l'intérieur du caisson de l'enceinte et de les restituer à l'avant de l'enceinte.
Avec ce système, lorsque les ondes situées à l’arrière sont en concordance de phase
avec celles émises a l’avant du haut-parleur, elles les amplifient. A l'inverse,
lorsqu’elles sont en opposition de phase,
ces ondes sont renvoyées à l’extérieur afin de réduire la résonance~\cite{US6275597}.
Cependant, ce système a pour inconvénient de provoquer dans certains cas
des bruits sonores dans l’event situé dans la partie inférieure du caisson.
Dans le cas du caisson ouvert, c'est la pièce entière qui sert de caisson. Cela convient mieux pour un twitter, mais peut provoquer des court-circuits acoustiques pour les basses fréquences~\cite{petoin} , d'où notre choix.