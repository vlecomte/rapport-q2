\section{Caissons}

\textbf{\textsc{À reformuler}}

Il existe sur le marché des haut-parleurs
différents types de caisson, tels que
les caisson basse-reflex, les caisson clos et encore bien d’autres.

Dans le choix du caisson de notre haut-parleur,
nous avons opté pour le caisson basse-reflex pour les graves
et un caisson ouvert pour les aigus.
Le premier s’explique par le fait que le caisson basse-reflex
contrairement à d’autres tel que le clos permet de récupérer
une partie des ondes se trouvant à l’arrière de la membrane.
Avec ce système, lorsque les ondes situées a l’arrière sont en phase
avec celles émises a l’avant du haut- parleur, elle les amplifie,
tandis que lorsqu’elles  sont opposées,
ces ondes sont renvoyées à l’extérieure afin de réduire la résonance.~\footnote{Issu du Brevet \textit{\og Loudspeaker system having a bass-reflex port (US6275597)\fg} de Roozen N.B., Hirschberg A. & Van Eck, P., \cite{US6275597}}
Cependant, ce système a pour inconvénient de provoquer dans certains cas
des bruits sonores dans l’évent, situe dans la partie inferieur du caisson.
En ce qui concerne le caisson ouvert,
le court-circuit acoustique ayant pour conséquence la perte du niveau
en dessous d’une certaine fréquence,
nous avons opté pour celui-ci qui réduit cet effet.
