\section{Filtres}
Nous avons choisi le thème "filtre" afin faire nos recherches bibliographiques. En effet ce terme nous semblait important, voire capital dans la bonne réalisation de ce projet, le cahier des charges nous imposant un réglage analogique des fréquences.


\paragraph{}
%Referencer le Kularathnal
Nous avons donc appris que les filtres sont dépendants de la fréquence et vice-versa. En effet ils permettent de laisser passer ou de bloquer certaines fréquences mais il y a pour ces mêmes filtres une "fréquence de coupure" qui limite le bon fonctionnement de ceux-ci. Nous nous sommes interessés au filtres basiques et 4 types en ressortent : passe-haut, passe-bas, passe-bande et coupe-bande. Dans notre cas, seuls les deux premiers types nous interessent. On découvre aussi dans nos sources que tous ces types peuvent être implanté dans un système de manière active, ou passive. La manière passive ne nécéssite pas d'alimentation externe mais bien de capacités et d'inductances de très hautes valeurs. Rajouter un amplificateur opérationnel nous amêne donc dans la méthode d'un filtre de type actif.

\paragraph{}
%Referencer l'Irwin
Une autre chose que nous avons découvert est une nouvelle manière de faire nos calculs dans les circuits, en utilisant les complexes. Cette méthode réduit drastiquement les calculs et nous a permit de gagner énormément de temps. En effet nous pouvons voire dans la section 4.3.3. l'utilisation des complexes dans la méthode de calcul des phaseurs.Le principal avantage étant d'éviter de résoudre une équation différentielle. Le calcul s'effectue comme pour les circuits de type DC, et donne en un seul nombre complexe, en même temps  l'amplitude et le déphasage d'un signal inconnu.