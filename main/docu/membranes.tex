\section{Membranes}

La séance d’information sur la recherche documentaire en S2
nous a aidé à déterminer les critères pouvant influencer
le bon fonctionnement de notre haut-parleur.
Parmi ceux-là, le choix de la membrane, qui constitue, avec la bobine mobile, la partie mobile
du haut-parleur.


\textbf{large{ICI rajouter la fonction/fonctionnement d'une membrane en qqs mots} et en profiter pour rajouter une référence}


La membrane est un composant complexe dépendant de nombreux paramètres. Le choix de sa forme, sa dimension et son matériau ainsi que son bon maintient ont une grande influence sur le rendu sonore. Un choix judicieux de ces critères permet d'optimiser le rendu sonore dans les aigus, les médiums ou les graves, selon ses désirs.

Tout d’abord, la membrane d’un haut-parleur artisanal
se doit d’être légère en masse et relativement rigide
car cela permet un déplacement plus facile et
sans risque d’endommager l’ensemble.
Le papier est donc, surprenamment, un matériau de choix.

Ensuite, la plupart des études faites jusqu’à ce jour,
ont montré que les membranes en forme de cône
donnent le son le plus puissant et de la meilleure qualité.
Preuve en est que cette forme est utilisé dans la quasi-totalité
des haut-parleurs sur le marché.
En particulier, la membrane devant être légère,
une forme non circulaire entrainera souvent des déformations non désirées,
diminuant ainsi le rendement sonore et, à terme, abîmera la membrane.

Le diamètre et la profondeur du cône de ces membranes
ont aussi un impact sur la bande de fréquences
qu’elles sont capables de reproduire fidèlement.
En effet, un cône possédant un petit diamètre et une petite profondeur
permet une meilleure réponse aux hautes fréquences,
tandis que les cônes de plus larges et plus profonds
seront plus adaptés aux fréquences basses et moyennes.
Pour cette raison, nous avons choisi de donner un diamètre
différent aux membranes de nos deux haut-parleurs,
utilisés pour isoler les graves et médiums dans un cas,
et les aigus dans l'autre.

Enfin, afin de reproduire des sons le plus fidèlement possible, il est important que la partie mobile du haut-parleur bouge de manière axiale par rapport au noyau.
Deux dispositifs sont prévus à cet effet : 
\begin{itemize}
\item un dispositif de centrage de la bobine mobile ou \textit{spider} qui \og a pour but d’empêcher la bobine mobile d’entrer en contact avec les parois de l’entrefer, sans toutefois gêner les mouvements utiles. Le spider idéal doit s’opposer à tout mouvement latéral de la bobine mobile ; il doit au repos la maintenir au centre de la zone de champ magnétique uniforme, ne pas exercer une résistance au mouvement augmentant avec l’amplitude et ne pas avoir de résonance propre.\fg \textbf{\large{citer encyclopédie Larousse}}
\item un dispositif de centrage et de suspension de la membrane, habituellement en \textbf{Large{trouver le matériau et en profiter pour rajouter réf}}, qui sert de raccord entre le saladier et la membrane. Il doit être le plus flexible possible, pour ne pas gêner le mouvement de la membrane. \textbf{\large{citer encyclopédie Larousse}}
\end{itemize}






