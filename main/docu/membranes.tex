\section{Membranes}

La séance d’information sur la recherche documentaire en S2
nous a aidé à déterminer les critères pouvant influencer
le bon fonctionnement de notre haut-parleur.
Parmi ceux-là, le choix de la membrane, qui consitue la partie mobile
du haut-parleur.

La membrane est un composant complexe
pour lequel le choix de nombreux paramètres
doit être effectué en fonction des contraintes particulières liées
à l'environnement dans lequel l'appareil fonctionne.

Tout d’abord, la membrane d’un haut-parleur artisanal
se doit d’être légère en masse et relativement rigide
car cela permet un déplacement plus facile et
sans risque d’endommager l’ensemble.
Le papier est donc, surprenamment, un matériau de choix.

Ensuite, la plupart des études faites jusqu’à ce jour,
ont montré que les membranes en forme de cône
donnent le son le plus puissant et de la meilleure qualité.
Preuve en est que cette forme est utilisé dans la quasi-totalité
des haut-parleurs sur le marché.
En particulier, la membrane devant être légère,
une forme non circulaire entrainera souvent des déformations non désirées,
diminuant ainsi le rendement sonore et, à terme, abîmant la membrane.

Enfin, le diamètre et la profondeur du cône de ces membranes
ont aussi un impact sur la bande de fréquences
qu’elles sont capables de reproduire fidèlement.
En effet, un cône possédant un petit diamètre et une petite profondeur
permet une meilleure réponse aux hautes fréquences,
tandis que les cônes de plus larges et plus profonds
seront plus adaptés aux fréquences basses et moyennes.
Pour cette raison, nous avons choisi de donner un diamètre
différent aux membranes de nos deux haut-parleurs,
et d'utiliser pour isoler les graves et médiums dans un cas,
et les aigus dans l'autre.
