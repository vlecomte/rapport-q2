\section{Filtres}
\Large{\textbf{En cours de correction par Giulia}}
Le second grand thème que nous avons décidé d'approfondir est le filtre. En effet, ce terme nous semblait 
important, voire capital pour la réalisation de ce projet, le cahier des charges nous imposant un réglage 
analogique des fréquences.

\paragraph{}
Comme le suggère son nom, un filtre laisse passer ou bloque certaines fréquences à partir d'une \emph{fréquence coupure}. Dans le cas d'un filtre idéal, la transition entre l'état passant et l'état bloquant est instantané. Mais dans la réalité, il existe une zone autour de la fréquence de coupure pour laquelle les fréquences sont seulement atténuées. Il existe quatre grandes catégories de filtres : 
\begin{itemize}
\item les filtres passe-haut : qui empêchent le passage des basses fréquences se situant sous la fréquence de coupure
\item les filtres passe-bas : ceux-ci 
\end{itemize}

Nous avons donc appris que les filtres sont dépendants de la fréquence et vice-versa. En effet ils permettent de 
laisser passer ou de bloquer certaines fréquences mais il y a pour ces mêmes filtres une "fréquence de coupure" 
qui limite le bon fonctionnement de ceux-ci. Nous nous sommes intéressés au filtres basiques et quatre types en 
ressortent : passe-haut, passe-bas, passe-bande et coupe-bande. Dans notre cas, seuls les deux premiers types nous 
intéressent. Le filtre passe-haut permet de réduire la bande de fréquences basses, et le filtre passe-bas permet de 
réduire les fréquences élevées. On découvre aussi dans nos sources que tous ces types peuvent être implanté dans un 
système de manière active, ou passive. La manière passive ne nécessite pas d'alimentation externe mais bien de 
capacités et d'inductances de très hautes valeurs. Rajouter un amplificateur opérationnel nous amène donc 
dans la méthode d'un filtre de type 
actif. \cite{Kularatna}

\paragraph{}
Notre deuxième grande découverte est une nouvelle manière de résoudre les circuits R-L-C, en utilisant les 
complexes \cite{Irwin}. Cette méthode réduit drastiquement les calculs et nous a permis de gagner 
énormément de temps. La section \ref{complexes} utilise cette méthode lors du calcul des phaseurs. Le principal 
avantage étant d'éviter de résoudre une équation différentielle. Le calcul s'effectue comme pour les circuits de 
type DC, et donne comme résultat un seul nombre complexe, dont nous pouvons déduire l'amplitude et le déphasage du 
signal inconnu.