\section{Filtres}
Nous avons choisi le thème "filtre" afin faire nos recherches bibliographiques. En effet ce terme nous semblait 
important, voire capital dans la bonne réalisation de ce projet, le cahier des charges nous imposant un réglage 
analogique des fréquences.

\paragraph{}
Nous avons donc appris que les filtres sont dépendants de la fréquence et vice-versa. En effet ils permettent de 
laisser passer ou de bloquer certaines fréquences mais il y a pour ces mêmes filtres une "fréquence de coupure" 
qui limite le bon fonctionnement de ceux-ci. Nous nous sommes interessés au filtres basiques et quatre types en 
ressortent : passe-haut, passe-bas, passe-bande et coupe-bande. Dans notre cas, seuls les deux premiers types nous 
interessent. Le filtre passe-haut permet de réduire la bande de fréquences basses, et le filtre passe-bas permet de 
réduire les fréquences élevées. On découvre aussi dans nos sources que tous ces types peuvent être implanté dans un 
système de manière active, ou passive. La manière passive ne nécéssite pas d'alimentation externe mais bien de 
capacités et d'inductances de très hautes valeurs. Rajouter un amplificateur opérationnel nous amène donc 
dans la méthode d'un filtre de type 
actif. \cite[p.~249-251]{Kularatna}

\paragraph{}
Notre deuxième grande découverte est une nouvelle manière de résoudre les circuits R-L-C, en utilisant les 
complexes \cite[p. 375-379]{Irwin}. Cette méthode réduit drastiquement les calculs et nous a permis de gagner 
énormément de temps. La section \ref{complexes} utilise cette méthode lors du calcul des phaseurs. Le principal 
avantage étant d'éviter de résoudre une équation différentielle. Le calcul s'effectue comme pour les circuits de 
type DC, et donne comme résultat un seul nombre complexe, dont nous pouvons déduire l'amplitude et le déphasage du 
signal inconnu.