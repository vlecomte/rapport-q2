\section{Filtres}
Le second grand thème que nous avons décidé d'approfondir est le filtre. En effet, ce terme nous semblait 
important, voire capital pour la réalisation de ce projet, le cahier des charges nous imposant un réglage 
analogique des fréquences.

\paragraph{}
Comme le suggère son nom, un filtre laisse passer ou bloque certaines fréquences à partir d'une \emph{fréquence coupure}. Dans le cas d'un filtre idéal, la transition entre l'état passant et l'état bloquant est instantané. Mais dans la réalité, il existe une zone autour de la fréquence de coupure pour laquelle les fréquences ne sont qu'atténuées. 

Il existe quatre grandes catégories de filtres \cite{Kularatna} : 
\begin{itemize}
\item les filtres passe-haut qui permettent de réduire la bande de fréquences basses;
\item les filtres passe-bas qui préservent les basses fréquences;
\item les filtres passe-bande qui ne laissent passer qu'un intervalle de fréquences situés entre une fréquence de coupure basse et une fréquence de coupure haute;
\item les filtres coupe-bande qui empêchent le passage d'une bande de fréquences.
\end{itemize}
Dans le cas de notre haut-parleur, nous avons combiné un filtre passe-haut et un passe-bas pour reproduire un filtre passe-bande. Les fréquences de coupure (haute et basse) -et donc la bande passante- sont réglables grâce à des potentiomètres.

Ces types de filtres peuvent être implantés dans un système de manière active ou passive. La manière passive ne nécessite pas d'alimentation externe mais bien des 
capacités et inductances de très hautes valeurs. Rajouter un amplificateur opérationnel nous amène dans la méthode d'un filtre de type actif. \cite{Kularatna}

\paragraph{}
Au cours de nos recherches, nous avons découvert une nouvelle manière de résoudre les circuits R-L-C, en utilisant les 
complexes \cite{Irwin}. Son principal avantage est d'éviter de résoudre des équations différentielles. Cette méthode réduit drastiquement les calculs et permet ainsi de gagner énormément de temps. Le calcul s'effectue comme pour les circuits de type DC, et donne comme résultat un seul nombre complexe, dont nous pouvons déduire l'amplitude et le déphasage du signal.
La section \ref{complexes} utilise cette méthode lors du calcul des phaseurs. 
