\subsection*{Introduction}

Les filtres servent à limiter les fréquences transmises au haut-parleur
à un certain intervalle, réglable par des potentiomètres.
Dans le schéma général du circuit (Fig.~\ref{fig:schema-gen-filtres}),
ils se situent après le réglage de volume, et avant l'amplification du signal.

Ces deux filtres (passe-haut et passe-bas) sont les éléments les plus complexes
du circuit, et ils ont constitué
une partie importante du travail de modélisation que nous avons effectué.
Nous y consacrons d'ailleurs trois annexes:
\begin{itemize}
    \item D'abord, une découverte de la méthode des phaseurs
        comme un isomorphisme
        entre les fonctions sinusoïdale et les complexes
        (Annexe~\ref{chap:phaseurs}).
    \item Ensuite,
        l'identification linéaire par morceaux du gain des filtres utilisés
        (Annexe~\ref{chap:approx-lin}).
    \item Enfin,
        la généralisation de cette identification à d'autres types de filtres
        (Annexe~\ref{chap:filtres-gen}).
\end{itemize}

Dans ce chapitre,
nous nous concentrerons plutôt sur le rôle pratique de ces filtres
dans notre circuit.

\subsubsection*{Plan du chapitre}
\begin{enumerate}
    \item Nous commencerons par expliquer plus en profondeur la \emph{fonction}
        des deux filtres de fréquence et leur effet dans la perception du son.
    \item Puis nous présenterons leurs \emph{caractéristiques techniques},
        c'est-à-dire les composants impliqués et leur agencement.
    \item Ensuite, nous ferons la \emph{modélisation} mathématique
        des deux filtres
        par la méthode des phaseurs et nous donnerons une interprétation
		des résultats obtenus.
    \item Après cela, nous justifierons le \emph{dimensionnement} par rapport
        aux fréquences de coupure.
    \item Enfin, nous ferons la synthèse des mesures effectuées ainsi qu'une
        \emph{confrontation} entre la théorie et la pratique.
\end{enumerate}
