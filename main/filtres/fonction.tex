\section{Fonction}

Comme expliqué dans l'introduction, les filtres servent à limiter
le domaine de fréquences du signal à un certain intervalle
$[f\ind{min},f\ind{max}]$.
Les fréquences situées en-dehors de cet intervalle seront atténuées,
comme l'illustre la figure~\ref{fig:filtre-ideal} en échelle log-log.

Ce genre de dispositif s'appelle un filtre \emph{passe-bande},
car il tend à laisser passer une bande de fréquences et exclure le reste.
Dans notre cas, il est formé par la combinaison d'un filtre
\emph{passe-haut} et d'un filtre \emph{passe-bas}.\cite{kanasewich1981time}

Le filtre \emph{passe-haut} laisse passer les hautes fréquences,
il se charge donc d'atténuer les basses fréquences ($f<f\ind{min}$);
tandis que le filtre \emph{passe-bas} laisse passer les basses fréquences,
il se charge donc d'atténuer les hautes fréquences ($f>f\ind{max}$).

L'oreille humaine est capable de
