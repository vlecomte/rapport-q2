\section{Caractéristiques techniques}
\label{sec:filtres/car-tech}

Comme illustré dans la figure~\ref{schema-elec-filtres},
la sous-ensemble des filtres est composé, dans l'ordre, de:
\begin{enumerate}
    \item un premier suiveur répétant le signal venant du réglage de volume
        à travers le JP\,1, grâce à l'ampli-op LM358N;
    \item le filtre passe-haut formé d'un
        condensateur de capacité $C_2 = 470\,\nano\farad$;
        et un potentiomètre de résistance
        maximale $R\ind{2,max} = 10\,\kilo\ohm$;
    \item un deuxième suiveur, sur le même ampli;
    \item le filtre passe-bas formé d'un potentiomètre de résistance
        maximale $R\ind{3,max} = 1\,\kilo\ohm$
        et un condensateur de même capacité $C_3 = 470\,\nano\farad$;
    \item la sortie au niveau du JP\,2 vers l'amplificateur audio.
\end{enumerate}

Les références des datasheets des composants se trouvent dans la bibliographie:
\begin{itemize}
    \item Ampli-op: Fairchild Semiconductor --- LM358N \cite{datasheet-lm358n};
    \item Capacités: AVX --- SR215E474MAR \cite{datasheet-470nf};
    \item Potentiomètres: Bourns --- 3386W-1-102LF et -103LF \cite{datasheet-pot}.
\end{itemize}

L'ampli-op et les capacités nous étaient imposés,
mais nous discuterons du choix des potentiomètres dans
la section~\ref{dimen-filtres} sur le dimensiomment.
