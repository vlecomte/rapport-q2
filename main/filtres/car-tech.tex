\section{Caractéristiques techniques}
\label{sec:filtres/car-tech}

\begin{figure}[h!]
 \centering
 \scalebox{0.7}{
  \begin{circuitikz}
   \draw
   (-2,0) node[op amp,yscale=-1] (opamp1) {}
   (opamp1.+) to [short, -o] (-3.5,0.5)
   node[anchor=south]{$V_1$}
   (opamp1.-) to [short] (-3.2,-1.5)
   to [short] (0,-1.5)
   to [short] (0,0)
   (-1,0) to [short, -*] (0,0)
   node[anchor=south]{$V_1$}
   to [C, l=$C_2$, -*] (3,0)
   (3,0.1) node[anchor=south]{$V_2$}
   (3,0) to [R, l=$R_2$] (3,-3)
   node [ground]{}
   (6,-0.5) node[op amp,yscale=-1] (opamp2) {}
   (opamp2.+) node[left ]{}
   (opamp2.-) node[left ]{} 
   (3,0) to [short] (opamp2.+)
   (4.80,-1) to [short] (4.80,-2)
   to [short] (8,-2)
   to [short, -*] (8,-0.5)
   to [short] (7,-0.5)
   (8,-0.4) node[anchor=south]{$V_2$}
   (8, -0.5) to [R, l=$R_3$] (11,-0.5)
   to [short] (11,-0.5)
   to [C, l=$C_3$] (11,-3.5)
   node[ground]{}
   (11,-0.5) to [short, -o] (12,-0.5)
   node[anchor=west]{$V_3$}
   ;
  \end{circuitikz}
  }
  \caption{Schéma électrique complet des filtres}
  \label{fig:schema-elec-filtres}
\end{figure}

Comme illustré dans la figure~\ref{fig:schema-elec-filtres},
la sous-ensemble des filtres est composé, dans l'ordre, de:
\begin{enumerate}
    \item un premier suiveur répétant le signal venant du réglage de volume
        à travers le JP\,1, grâce à l'ampli-op LM358N;
    \item le filtre passe-haut formé d'un
        condensateur de capacité $C_2 = 470\,\nano\farad$;
        et un potentiomètre de résistance
        maximale $R\ind{2,max} = 10\,\kilo\ohm$;
    \item un deuxième suiveur, sur le même ampli;
    \item le filtre passe-bas formé d'un potentiomètre de résistance
        maximale $R\ind{3,max} = 1\,\kilo\ohm$
        et un condensateur de même capacité $C_3 = 470\,\nano\farad$;
    \item la sortie au niveau du JP\,2 vers l'amplificateur audio.
\end{enumerate}

Les références des datasheets des composants se trouvent dans la bibliographie:
\begin{itemize}
    \item Ampli-op: Fairchild Semiconductor --- LM358N \cite{datasheet-lm358n};
    \item Capacités: AVX --- SR215E474MAR \cite{datasheet-470nf};
    \item Potentiomètres: Bourns --- 3386W-1-102LF et -103LF \cite{datasheet-pot}.
\end{itemize}

L'ampli-op et les capacités nous étaient imposés,
mais nous discuterons du choix des potentiomètres dans
la section~\ref{sec:filtres/dimens} sur le dimensiomment.
