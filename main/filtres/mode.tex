\section{Modélisation}

Dans cette section, nous allons commencer par analyser les filtres
par une bonne intuition,
ensuite nous poserons nos hypothèses de modélisation,
puis nous trouverons les fonctions de transfert et en extrairons
des tendances générales
et les expressions des gains et amplitudes,
et enfin nous interpréterons les résultats.

\subsection{Intuition}

Le filtre passe-haut comme le filtre passe-bas
consistent en une capacité et une résistance en série,
avec une tension forcée aux bouts.

Prenons une analogie hydraulique:
les tensions deviennent des différences de pressions
et les courants deviennent des débits.
On peut voir la capacité comme une large membrane élastique
tendue par les déplacements d'eau;
et la résistance comme un fin tuyau
qui requiert une plus grande pression pour un même débit.
La source est une pompe qui crée une différence de pression.

Les deux représentations sont comparées dans la figure~\ref{fig:ana-hydr}.

Pour de basses fréquences, la pression oscille doucement.
La membrane de la capacité à donc bien le temps de s'équilibrer avec elle,
tandis que la pression à travers le fin tuyau sera petite.

Pour de hautes fréquences, au contraire,
la pression oscille très rapidement.
La membrane n'a donc pas le temps d'être tendue par le débit,
et toute la différence de pression
se trouvera aux extrémités du fin tuyau.

En conclusion, on s'attend à ce que le filtre passe-haut,
qui prend la tension aux bornes de la résistance (fin tuyau),
atténue les basses fréquences et conserve les hautes fréquences;
et à ce que le filtre passe-bas,
qui prend la tension aux bornes de la capacité (membrane),
conserve les basses fréquences et atténue les hautes fréquences.


\subsection{Hypothèses}

Pour modéliser la réponse des filtres à différents signaux nous allons utiliser
la méthode des phaseurs, comme décrite dans l'annexe~\ref{chap:phaseurs}.
Nous supposerons donc ici que le signal est une sinusoïde pure, les autres cas
pouvant s'y ramener en prenant la transformée de Fourier.

Nous supposons également que tous les composants et les liaisons entre ceux-ci
sont idéaux.
En particulier, nous considérons que les courants d'entrée de l'ampli-op
et l'ampli audio sont négligeable.

\subsection{Résolution}

Définissons comme sur la figure~\ref{fig:filtres-v1v2v3}
$v_1(t),v_2(t),v_3(t)$ les tensions
à l'entrée, après le filtre passe-haut et après le filtre passe-bas;
et $\overline{V_1}, \overline{V_2}, \overline{V_3}$ leurs phaseurs associés.

Tout le courant passant à travers le condensateur $C_2$ passe
aussi à travers la résistance $R_2$.
Par conséquent, on peut appliquer la loi de division des tensions,
avec les impédances $\frac{1}{j\omega C_2}$ et $R_2$:
\begin{equation}
    \overline{V_2} = \frac{R_2}{\frac{1}{j\omega C_2} + R_2}\,\overline{V_1}
    = \frac{j\omega R_2C_2}{1+j\omega R_2C_2}\,\overline{V_1}
\end{equation}
La fonction de transfert du filtre passe-haut est donc
$H\ind{ph}(j\omega) = \frac{j\omega R_2C_2}{1+j\omega R_2C_2}$.

Pour le filtre passe-bas, on peut supposer de manière similaire que
le courant à l'entrée de l'ampli audio est négligeable,
et appliquer la loi de division des tensions avec les impédances
$R_3$ et $\frac{1}{j\omega C_3}$:
\begin{equation}
    \overline{V_3} =
    \frac{\frac{1}{j\omega C_3}}{\frac{1}{j\omega C_3} + R_3}\,\overline{V_2}
    = \frac{1}{1+j\omega R_3C_3}\,\overline{V_2}
\end{equation}
Le gain en tension du filtre passe-bas est donc
$H\ind{pb}(j\omega) = \frac{1}{1 + j\omega R_3C_3}$.

Étant donné que la tension $v_2$ est répétée de la sortie du filtre passe-haut
à l'entrée du filtre passe-bas, la fonction de transfert totale du système est:
\begin{align}
    H\ind{tot}(j\omega) &= H\ind{ph}(j\omega)\,H\ind{pb}(j\omega) =
    \frac{j\omega R_2C_2}{1 + j\omega R_2C_2}\,\cdot\,
    \frac{1}{1 + j\omega R_3C_3}\\
    &= \frac{j\omega R_2C_2}{1 + j\omega\,(R_2C_2+R_3C_3) +
        (j\omega)^2\,(R_2C_2R_3C_3)}
\end{align}

\subsection{Tendances}

\textbf{\textsc{À bouger en annexe}}

Avant de calculer les gains et déphasages en détail,
nous allons mettre en évidence des tendances dans les fonctions de transfert.

Commençons par le filtre passe-haut, dont la fonction de transfert est
$H\ind{ph}(j\omega) = \frac{j\omega R_2C_2}{1+j\omega R_2C_2}$.
Le dénominateur contient deux termes qui «dominent» à tour de rôle:
le premier, 1, sera le plus grand en norme pour de petites valeurs de $\omega$,
jusqu'au point $\omega = 1/R_2C_2$, où $j\omega R_2C_2$, qui grandit
avec $\omega$, «prendra le relais».
Notons cette pulsation critique $\omega\ind{bas}= 1/R_2C_2$ et
appelons-la \emph{pulsation de coupure}.

Plus précisément,
\begin{itemize}
    \item quand $\omega << \omega\ind{bas}$,
        $H\ind{ph} \approx \frac{j\omega R_2C_2}{1} = j\omega R_2C_2$;
    \item quand $\omega >> \omega\ind{bas}$,
        $H\ind{ph} \approx \frac{j\omega R_2C_2}{j\omega R_2C_2} = 1$.
\end{itemize}

Cette rapide analyse confirme notre intuition et la précise:
\begin{itemize}
    \item pour de petites pulsations,
        le signal est de moins en moins atténué et est en avance de $\pi/2$
        sur l'entrée (à cause du facteur $j$);
    \item à partir de la pulsation de coupure le signal sera presque
        restitué à l'identique.
\end{itemize}

Pour le filtre passe-bas, la situation est analogue.
Le dénominateur fait apparaître une pulsation de coupure
$\omega\ind{haut} = 1/R_3C_3$.
À partir de la fonction de transfert
$H\ind{pb}(j\omega) = \frac{1}{1 + j\omega R_3C_3}$,
on découvre que:
\begin{itemize}
    \item quand $\omega << \omega\ind{haut}$, $H\ind{pb} \approx 1$;
    \item quand $\omega >> \omega\ind{haut}$, $H\ind{pb}
        \approx \frac{1}{j\omega R_3C_3}$;
\end{itemize}
ce qui confirme et précise notre intuition:
\begin{itemize}
    \item pour de petites pulsations, le signal sera presque restitué
        à l'identique;
    \item à partir de la pulsation de coupure, le signal est de plus en plus
        atténué
        et est en retard de $\pi/2$ sur l'entrée (à cause du facteur $1/j$).
\end{itemize}

\subsection{Gain et déphasage}

Intéressons-nous maintenant de manière plus rigoureuse
au \emph{gain} (rapport d'amplitude) et au  \emph{déphasage} (décalage)
entre l'entrée et la sortie.
Ils correspondent respectivement au module et à l'argument de la fonction de
transfert.

C'est-à-dire, pour le filtre passe-haut:
\begin{align}
    G\ind{ph}(j\omega) &= |H\ind{ph}(j\omega)|
    = \frac{|j\omega R_2C_2|}{|1 + j\omega R_2C_2|}
    = \frac{\omega R_2C_2}{\sqrt{1+(\omega R_2C_2)^2}}\\
    \phi\ind{ph}(j\omega) &= \arg\big(H\ind{ph}(j\omega)\big)
    %= \arctan(j\omega R_2C_2) - \arctan(1+j\omega R_2C_2)
    = \arctan\left(\frac{1}{\omega R_2C_2}\right)
\end{align}
et pour le filtre passe-bas:
\begin{align}
    G\ind{pb}(j\omega) &= |H\ind{pb}(j\omega)|
    = \frac{|1|}{|1 + j\omega R_3C_3|}
    = \frac{1}{\sqrt{1+(\omega R_3C_3)^2}}\\
    \phi\ind{pb}(j\omega) &= \arg\big(H\ind{pb}(j\omega)\big)
    = \arctan(-\omega R_3C_3)
\end{align}

Ces résultats sont illustrés par des graphes dans
les figures~\ref{fig:graphes-pb} et~\ref{fig:graphes-pb}, avec en abcisse la pulsation
et en ordonnée le gain et le déphasage.
Les tendances découvertes sont en pointillés.
Ici, les potentiomètres sont tous deux réglés à $20\%$,
donc $R_2 = 200\,\ohm$ et $R_3 = 2\,\kilo\ohm$.

Nous utilisons une échelle logarithmique pour les fréquences
ou pulsations, ainsi que pour les gains.
Les raisons de ce choix sont expliquées en détail dans
la section~\ref{sec:log}.

Enfin, pour la combinaison des deux filtres,
les fonctions de transfert sont multipliées,
donc les gains sont multipliés et les déphasages sont additionnés.
Le résultat est illustré dans la figure~\ref{fig:graphes-bande}.

\subsection{Interprétation}

\textbf{\textsc{À réécrire et compléter}}

Nous pouvons voir dans la figure~\ref{fig:graphes-bande}
que notre combinaison de filtres a bien le résultat attendu.
En effet, on voit que les signaux dont les pulsations sont dans l'intervalle
$[\omega\ind{bas},\omega\ind{haut}]$
\footnote{Ou dont les fréquences sont dans l'intervalle
    $[f\ind{bas},f\ind{haut}]$ avec
    $f\ind{bas} = \omega\ind{bas}/2\pi$ et
    $f\ind{haut} = \omega\ind{haut}/2\pi$.}
sont conservées.

La première remarque à faire est que les filtres ne sont pas des filtres idéaux
en ce sens que la coupure de signal n'est pas brutale:
dans le filtre passe-haut comme dans le filtre passe-bas,
l'amplitude décroit linéairement en échelle log-log.
Cela dit, ce n'est pas forcément une mauvaise chose.
En effet, cela permet d'atténuer certaines fréquences sans pour autant
les supprimer complètement.
