\subsection{Intuition}

Le filtre passe-haut comme le filtre passe-bas
consistent en une capacité et une résistance en série,
avec une tension forcée aux bouts.

Prenons une analogie hydraulique:
les tensions deviennent des différences de pressions
et les courants deviennent des débits.
On peut voir la capacité comme une large membrane élastique
tendue par les déplacements d'eau;
et la résistance comme un fin tuyau
qui requiert une plus grande pression pour un même débit.
La source est une pompe qui crée une différence de pression.

Les deux représentations sont comparées dans la figure~\ref{fig:ana-hydr}.

Pour de basses fréquences, la pression oscille doucement.
La membrane de la capacité à donc bien le temps de s'équilibrer avec elle,
tandis que la pression à travers le fin tuyau sera petite.

Pour de hautes fréquences, au contraire,
la pression oscille très rapidement.
La membrane n'a donc pas le temps d'être tendue par le débit,
et toute la différence de pression
se trouvera aux extrémités du fin tuyau.

En conclusion, on s'attend à ce que le filtre passe-haut,
qui prend la tension aux bornes de la résistance (fin tuyau),
atténue les basses fréquences et conserve les hautes fréquences;
et à ce que le filtre passe-bas,
qui prend la tension aux bornes de la capacité (membrane),
conserve les basses fréquences et atténue les hautes fréquences.
