\subsection{Résolution}
\label{complexes}

Définissons comme sur la figure~\ref{fig:filtres-v1v2v3}
$V_1$, $V_2$, et $V_3$ les phaseurs des tensions
à l'entrée, après le filtre passe-haut, et après le filtre passe-bas.
Pour une explication des phaseurs, voir l'annexe~\ref{chap:phaseurs},
et plus particulièrement la section~\ref{sec:phaseurs/impedance}.

Tout le courant passant à travers le condensateur $C_2$ passe
aussi à travers la résistance $R_2$.
Par conséquent, on peut appliquer la loi de division des tensions,
avec les impédances $\frac{1}{j\omega C_2}$ et $R_2$:
\begin{equation}
    V_2\,/\,V_1
    = \frac{R_2}{\frac{1}{j\omega C_2} + R_2}
    = \frac{j\omega R_2C_2}{1+j\omega R_2C_2}
\end{equation}
Ce rapport est appelé \emph{fonction de transfert}, et nous le notons
$H\ind{ph}(j\omega)$.

Pour le filtre passe-bas, de la même manière, tout le courant
passant à travers la résistance $R_3$ passe aussi à travers le
condensateur $C_3$, donc on peut appliquer
la loi de division des tensions avec les impédances
$R_3$ et $\frac{1}{j\omega C_3}$:
\begin{equation}
    V_3\,/\,V_2
    = \frac{\frac{1}{j\omega C_3}}{\frac{1}{j\omega C_3} + R_3}
    = \frac{1}{1+j\omega R_3C_3}
\end{equation}
que nous notons $H\ind{pb}(j\omega)$.

Étant donné que la tension $V_2$ est répétée de la sortie du filtre passe-haut
à l'entrée du filtre passe-bas, la fonction de transfert totale du système est:
\begin{align}
    H\ind{tot}(j\omega) &= H\ind{ph}(j\omega)\,H\ind{pb}(j\omega) =
    \frac{j\omega R_2C_2}{1 + j\omega R_2C_2}\,\cdot\,
    \frac{1}{1 + j\omega R_3C_3}\\
    &= \frac{j\omega R_2C_2}{1 + j\omega\,(R_2C_2+R_3C_3) +
        (j\omega)^2\,(R_2C_2R_3C_3)}
\end{align}
