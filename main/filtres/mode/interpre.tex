\subsection{Interprétation}

\textbf{\textsc{À réécrire et compléter}}

Nous pouvons voir dans la figure~\ref{fig:graphes-bande}
que notre combinaison de filtres a bien le résultat attendu.
En effet, on voit que les signaux dont les pulsations sont dans l'intervalle
$[\omega\ind{bas},\omega\ind{haut}]$
\footnote{Ou dont les fréquences sont dans l'intervalle
    $[f\ind{bas},f\ind{haut}]$ avec
    $f\ind{bas} = \omega\ind{bas}/2\pi$ et
    $f\ind{haut} = \omega\ind{haut}/2\pi$.}
sont conservées.

La première remarque à faire est que les filtres ne sont pas des filtres idéaux
en ce sens que la coupure de signal n'est pas brutale:
dans le filtre passe-haut comme dans le filtre passe-bas,
l'amplitude décroit linéairement en échelle log-log.
Cela dit, ce n'est pas forcément une mauvaise chose.
En effet, cela permet d'atténuer certaines fréquences sans pour autant
les supprimer complètement.
