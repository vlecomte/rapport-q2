\subsection{Interprétation}

D'abord, nous pouvons voir dans la figure~\ref{fig:graphes-bande}
que notre combinaison de filtres répond bien à la définition de
filtre passe-bande.
En effet, les signaux dont les pulsations sont dans l'intervalle
$[\omega\ind{bas},\omega\ind{haut}]$
\footnote{
    Ou dont les fréquences sont dans l'intervalle
    $[f\ind{bas},f\ind{haut}]$ avec
    $f\ind{bas} = \omega\ind{bas}/2\pi$ et
    $f\ind{haut} = \omega\ind{haut}/2\pi$.
}
sont conservés, et les autres signaux sont atténués.

Ce n'est par contre pas un filtre passe-bande parfait:
au lieu d'une coupure nette de signal, le gain décroît
de manière linéaire en échelle log-log.
Cela peut avoir des avantages.
En effet, cela permet d'atténuer certaines fréquences sans pour autant
les supprimer complètement.

Enfin, les pulsations de coupures $1/R_2C_2$ et $1/R_3C_3$
peuvent être réglées facilement en changeant les valeurs des résistances,
grâce au curseur mobile des potentiomètres.
Dans la prochaine section, nous sélectionnerons les potentiomètres
qui permettront le choix de pulsations de coupure
le plus adapté.
