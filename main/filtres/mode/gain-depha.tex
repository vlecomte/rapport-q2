\subsection{Gain et déphasage}

Nous allons maintenant déterminer
le \emph{gain} (rapport d'amplitude) et le  \emph{déphasage} (décalage)
entre l'entrée et la sortie de chacun des filtres.
Ils correspondent respectivement au module et à l'argument de la fonction de
transfert.

C'est-à-dire, pour le filtre passe-haut:
\begin{align}
    G\ind{ph}(j\omega) &= |H\ind{ph}(j\omega)|
    = \frac{|j\omega R_2C_2|}{|1 + j\omega R_2C_2|}
    = \frac{\omega R_2C_2}{\sqrt{1+(\omega R_2C_2)^2}}\\
    \phi\ind{ph}(j\omega) &= \arg\big(H\ind{ph}(j\omega)\big)
    %= \arctan(j\omega R_2C_2) - \arctan(1+j\omega R_2C_2)
    = \arctan\left(\frac{1}{\omega R_2C_2}\right)
\end{align}
et pour le filtre passe-bas:
\begin{align}
    G\ind{pb}(j\omega) &= |H\ind{pb}(j\omega)|
    = \frac{|1|}{|1 + j\omega R_3C_3|}
    = \frac{1}{\sqrt{1+(\omega R_3C_3)^2}}\\
    \phi\ind{pb}(j\omega) &= \arg\big(H\ind{pb}(j\omega)\big)
    = \arctan(-\omega R_3C_3)
\end{align}

Ces résultats sont illustrés par des graphes dans
les figures~\ref{fig:graphes-pb} et~\ref{fig:graphes-pb},
avec en abcisse la pulsation
et en ordonnée le gain et le déphasage.
Les tendances découvertes sont en pointillés.
Ici, les potentiomètres sont tous deux réglés à $20\%$,
donc $R_2 = 200\,\ohm$ et $R_3 = 2\,\kilo\ohm$.

Nous utilisons une échelle logarithmique pour les fréquences
ou pulsations, ainsi que pour les gains.
Les raisons de ce choix sont expliquées en détail dans
la section~\ref{sec:log}.

Enfin, pour la combinaison des deux filtres,
les fonctions de transfert sont multipliées,
donc les gains sont multipliés et les déphasages sont additionnés.%
\footnote{
    Cela découle des propriétés du module et de l'argument.
    En effet, pour des complexes $x$ et $y$,
    on a $|xy| = |x||y|$ et $\arg(xy) = \arg x + \arg y$.
}
Le résultat est illustré dans la figure~\ref{fig:graphes-bande}.

Nous avons noté sur les graphes les pulsations
$\omega\ind{bas} = 1/R_2C_2$ et $\omega\ind{haut}=1/R_3C_3$,%
\footnote{Nous omettrons les parenthèses lorsque cela améliore la lisibilité.}
appelées \emph{pulsations de coupure}.
Elles indiquent le passage d'un comportement à l'autre.
Nous en expliquerons l'origine de ces valeurs en détail dans les
annexes~\ref{chap:approx-lin} et~\ref{chap:filtres-gen}.
