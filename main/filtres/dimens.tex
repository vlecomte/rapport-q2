\section{Dimensionnement}
\label{sec:filtres/dimens}

Comme expliqué dans la section~\ref{sec:filtres/car-tech}
sur les caractéristiques techniques,
le seul choix que nous avons eu à faire est celui des potentiomètres.
Mais c'est également le plus critique, car
il détermine les fréquences de coupure qui seront accessibles à l'utilisateur.

Les filtres passe-haut sont, comme les filtres passe-bas,
constitués d'une résistance $R$ et d'une capacité $C$. Pour ceux-ci,
la pulsation de coupure est $1/RC$
et la fréquence de coupure est $1/(2\pi RC)$.

Si l'on peut faire varier la résistance de $0$ à une valeur $R\ind{max}$,
alors la fréquence de coupure peut varier dans l'intervalle
$[1/(2\pi R\ind{max}C),\infty)$,
c'est-à-dire que l'on peut décaler la courbe
de réponse arbitrairement vers la droite.
Toutefois, nous considérons qu'en-dessous d'un réglage à $10\%$,
l'erreur relative est trop grande pour choisir correctement la fréquence,
ce qui limite l'intervalle à $[1/(2\pi R\ind{max}C),10/(2\pi R\ind{max}C)]$.

Les capacités imposées avaient une valeur $C=470\,\nano\farad$,
et pour les résistances nous avions le choix entre
$R\ind{max}=10\,\kilo\ohm$, $1\,\kilo\ohm$ ou $100\,\ohm$.
Elles donnaient donc respectivement les intervalles
$[34\,\hertz,340\,\hertz]$, $[340\,\hertz,3.4\,\kilo\hertz]$
et $[3.4\,\kilo\hertz,34\,\kilo\hertz]$
pour la sélection de la fréquence de coupure.

Le facteur décisif de notre choix a été
qu'une fréquence de $3.4\,\kilo\hertz$ est
presque deux octaves plus haute que la note la plus aiguë
qu'une soprano peut atteindre (A5, ou $880\,\hertz$).%
\cite{vocal-ranges}
Dès lors, pouvoir régler la fréquence de coupure d'un filtre
entre $3.4\,\kilo\hertz$ et $34\,\kilo\hertz$,
avec la résistance $R\ind{max} = 100\,\ohm$, aura très peu d'effet.

Ensuite, nous avons attribué les deux potentiomètres restants
aux deux filtres, sachant que la fréquence de coupure du passe-haut
doit être plus basse que celle du passe-bas,
ce qui a donné $R\ind{2,max}=10\,\kilo\ohm$ pour le passe-haut,
et $R\ind{3,max}=1\,\kilo\ohm$ pour le passe-bas.
