\section{Confrontation avec la pratique}
\label{sec:filtres/confront}

Afin de valider notre modélisation,
nous avons fait des mesures sur un circuit RC
des tensions à l'entrée, à travers la résistance
et à travers la capacité pour différentes fréquences.
Cela nous a permis de mesurer en un seul circuit
les filtres passe-haut et passe-bas.
Les résultats sont présentés et illustrés
dans la section~\ref{sec:mesures/filtres-1k}.

À l'aide de l'une des méthodes mathématiques présentées dans
l'annexe~\ref{chap:approx-lin},
nous avons dégagé dans les courbes des gains en échelle log-log
des tendances linéaires.
Celles-ci sont illustrées dans la figure~\ref{fig:ph-pb-tendances}.
Les points de mesure épousent presque parfaitement les droites,
ce qui est un premier signe que la théorie rejoint la pratique.

Ensuite, nous avons déterminé la fréquence de coupure,
qui se trouve à l'intersection des deux droites.
Pour rappel, l'expression théorique de la fréquence de coupure est
$f_c=1/(2\pi RC)$, ce qui pour les valeurs mesurées de résistance et
de capacité donne $f_c=366.46\,\hertz$.

Les valeurs obtenues pour le filtre passe-haut et le filtre passe-bas
sont respectivement $325.12\,\hertz$ et $317.51\,\hertz$,
soit $11\,\%$ et $13\,\%$ plus bas que la valeur théorique.

Une différence si faible est facilement explicable par des imprécisions
de mesure et des capacités parasites au niveau des contacts,
le circuit de test n'étant pas soudé.
Par conséquent, nous considérons que la théorie a sufisamment de preuves
pour être validée.
