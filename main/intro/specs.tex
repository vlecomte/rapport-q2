%Permet de changer les marges temporairement
\newenvironment{changemargin}[2]{\begin{list}{}{%
\setlength{\topsep}{0pt}%
\setlength{\leftmargin}{0pt}%
\setlength{\rightmargin}{0pt}%
\setlength{\listparindent}{\parindent}%
\setlength{\itemindent}{\parindent}%
\setlength{\parsep}{0pt plus 1pt}%
\addtolength{\leftmargin}{#1}%
\addtolength{\rightmargin}{#2}%
}\item }{\end{list}}
%

\section{Spécifications}

Afin de mener a bien la conception de notre haut-parleur, il nous a fallu tout d’abord élaborer un cahier des 
charges listant l’ensemble des conditions que devrait remplir, à terme, ce dernier. Le cahier des charges reprends 
les contraintes, performances ainsi que fonctions de notre haut-parleur. Ci-dessous, vous retrouverez la version 
finale de notre cahier des charges. 

\begin{table}[H]
\begin{changemargin}{-1cm}{2cm}
\begin{tabular}{|p{2cm}|p{1.5cm}|p{13cm}|}

 \hline 
 \bsc{Gr. 11.32} & \multicolumn{2}{r|} {12/03/2014 - Version 4}\\
 \cline{1-1}
 
	 	\multicolumn{3}{|c|}{\large{\textbf{Cahier des Charges d'un haut-parleur}}}\\
	\multicolumn{3}{|l|}{}\\
	
 		\multicolumn{3}{|p{16.5cm}|}{\textbf{Contexte :} Dans le cadre du cours de projet, il nous a été demandé de modéliser, construire, faire fonctionner et effectuer des tests sur un haut-parleur pouvant être relié à un smartphone. }\\
	\multicolumn{3}{|l|}{}\\
 \hline
 
 		\textbf{Date} & \textbf{Origine} & \\
 \hline
 	& &\\
		& & \textbf{Fonctions Principales} \\
 
		05-02-2014 & client & \textbf{FP 1 :} Le haut-parleur est capable d’amplifier un signal électrique provenant d’un smartphone ou d’un baladeur MP3.\\

		05-02-2014 & client & \textbf{FP 2 :} Le haut-parleur est en mesure de transformer un signal électrique en un signal sonore afin de le reproduire fidèlement.  \\
	
		05-02-2014 & client & \textbf{FP 3 :} Le volume du son est réglable.\\
	
		05-02-2014 & client & \textbf{FP 4 :} Le signal sonore est réglable en aigus et en graves.\\
	
		05-02-2014 & client & \textbf{FP 5 :} La puissance maximale du haut-parleur est de $2,5$ \watt.\\
	& &\\
\hline
	 & &\\
	 	& & \textbf{Critères et niveaux des Fonctions Principales} \\

	 	12-03-2014 & groupe & \textbf{C 4.1} : La fréquence maximale peut être réglée entre $340\,\hertz$ et $3400\,\hertz$ et la fréquence minimale entre $34\,\hertz$ et $340\,\hertz$.\\
	 & &\\
\hline
	 & &\\
	 & & \textbf{ Fonctions de Contraintes}\\

		05-02-2014 & client & \textbf{FC 6 :} La prise d’entrée est définie.\\

		05-02-2014 & groupe & \textbf{FC 7 :} La température de fonctionnement est prédéfinie.\\
	
		05-02-2014 & client & \textbf{FC 8 :} Le type d’alimentation du haut-parleur est imposé.\\
	 & &\\
\hline
	& &\\
		& & \textbf{ Critères et niveaux des Fonctions de Contraintes}\\

	 	05-02-2014 & client & \textbf{C 6.1 :} Le signal entre via une prise Jack TRS $3.5$ \milli \meter.\\

		05-02-2014 & groupe & \textbf{C 7.1 :} Le haut-parleur est opérationnel dans des températures comprises entre $0$ et $40\,\celsius$.\\
	
		05-02-2014 & groupe & \textbf{C 8.1 :} Le haut-parleur est alimenté par une source de tension continue et réglable de maximum $30$ \volt.\\
	
\hline
 \end{tabular}
 \end{changemargin}
  \end{table}