
\documentclass[a4paper, 11pt]{article}

 \usepackage[utf8x]{inputenc}
 \usepackage[T1]{fontenc} 
 \usepackage[french]{babel} 
 \usepackage{lmodern}
 \usepackage{multirow}
 \usepackage{tabularx}
 
 \usepackage{SIunits}
 
 
 \usepackage[top=2cm, bottom=2cm, left=1cm, right=1cm]{geometry}
 
 \begin{document}
\begin{tabular}{|p{2cm}|p{1.5cm}|p{13.2cm}|}

 \hline 
 \bsc{Groupe 11.32} & \multicolumn{2}{r|} {Date : 12/03/2014}\\
  &  \multicolumn{2}{r|}{Version : 4} \\
 \hline 
	 \multicolumn{3}{|l|}{}\\
	 \multicolumn{3}{|l|}{}\\
	 	\multicolumn{3}{|c|}{\Large{\textbf{Cahier des Charges d'un haut-parleur}}}\\
	\multicolumn{3}{|l|}{}\\
	
 		\multicolumn{3}{|p{16.5cm}|}{\large{\textbf{Contexte :} Dans le cadre du cours de projet, il nous a été demandé de modéliser, construire, faire fonctionner et effectuer des tests sur un haut-parleur pouvant être relié à un smartphone. }}\\
	\multicolumn{3}{|l|}{}\\
 \hline
 
 		\textbf{Date} & \textbf{Origine} & \\
 \hline
 	& &\\
		& & \textbf{Fonctions Principales} \\
 	& &\\
		05-02-2014 & client & \textbf{FP 1 :} Le haut-parleur est capable d’amplifier un signal électrique provenant d’un Smartphone ou d’un baladeur MP3.\\
	& &\\
		05-02-2014 & client & \textbf{FP 2 :} Le haut-parleur est en mesure de transformer un signal électrique en un signal sonore afin de le reproduire fidèlement.  \\
	& &\\
		05-02-2014 & client & \textbf{FP 3 :} Le volume du son est réglable.\\
	& &\\
		05-02-2014 & client & \textbf{FP 4 :} Le signal sonore est réglable en aigus et en graves.\\
	& &\\
		05-02-2014 & client & \textbf{FP 5 :} La puissance maximale du haut-parleur est de $2,5  \watt$.\\
	& &\\
\hline
	 & &\\
	 	& & \textbf{Critères et niveaux des Fonctions Principales} \\
	 & &\\
	 	12-03-2014 & groupe & \textbf{C 1.1} : La fréquence maximale peut être réglée entre $340$ Hz et $6800$ Hz et la fréquence minimale entre $34$ Hz et $340$ Hz.\\
	 & &\\
\hline
	 & &\\
	 & & \textbf{ Fonctions de Contraintes}\\
	 & &\\
		05-02-2014 & client & \textbf{FC 4 :} La prise d’entrée est définie.\\
	 & &\\
		05-02-2014 & groupe & \textbf{FC 5 :} La température de fonctionnement est prédéfinie.\\
	 & &\\
		05-02-2014 & client & \textbf{FC 6 :} Le type d’alimentation du haut-parleur est imposé.\\
	 & &\\
\hline
	& &\\
		& & \textbf{ Critères et niveaux des Fonctions de Contraintes}\\
	& &\\
	 	05-02-2014 & client & \textbf{C 4.1 :} Le signal entre via une prise Jack TRS $3.5$ mm.\\
	& &\\
		05-02-2014 & groupe & \textbf{C 5.1 :} Le haut-parleur est opérationnel dans des températures comprises entre $0$ et $40$°C.\\
	& &\\
		05-02-2014 & groupe & \textbf{C 6.1 :} Le haut-parleur est alimenté par une source de tension continue et réglable de maximum de $30$ \volt.\\
	 & &\\
\hline
 \end{tabular}
 \end{document}
 





